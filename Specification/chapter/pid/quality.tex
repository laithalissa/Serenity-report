\section{Quality Controls}
\label{section:quality}

Quality controls are a set of methods that allow the product to be tested against the specification, identifying cases in which the product will not meet the specification.
Quality control can be automated for requirements that test the behaviour of the product, such as functional requirements. Non-functional requirements, which specify the the qualities that are required for the project to achieve a desired behaviour, are generally not quantifiable and therefore cannot be automated.

% what quality control is
% what it will do for us


% meets the specification
% bug free
% 

\subsection{Unit Testing}
Unit tests are used to test small, individual units of source code and ensure that the
code meets its intended design. Unit testing is a very useful practice because it helps
catch errors in code and allows a developer to refactor code by ensuring that it continues
to work as before.

Unit testing is very well supported in Haskell. There is a library called \emph{QuickCheck} that can take advantage of referential transparency in order to create random test instances. There is also more standard unit test support with \emph{HUnit}. Together these allow easy to write, highly effective testing strategies.

\subsection{Component Testing}
Component testing is similar to unit testing, but focuses on larger pieces of the source
code; it is used to test the integration of a number of units. Component is useful to
ensure that the small units of code, which, through unit testing, are known to work
individually, work together correctly.

\subsection{Continuous Integration}
Unit tests are only really useful if they are run regularly. Doing so allows developers
to catch bugs as soon as they are introduced. This is useful because, as van Emden
and Moonen note, the cost of fixing a bug is much lower if it is discovered earlier in
the development cycle.\cite{emden2002} Therefore, a continuous integration server
will be used to perform a full build and test of the software every time a new change
is pushed to the source control repository. If the test suite fails then project members
will be notified via email. This means that it is easy to see which change caused a
failure and the team can quickly fix the problem.

\subsection{Playtesting}
%Running the game, with a playtester playing the game, looking for incorrections in the semantics of the game.
%At this level, it is generally considered impossible to automate the playtesting stage, hence an actual person is rquired.

% major: fine tuning the game, minor: testing the gameplay meets the spec, 
% feedback

% its about fixing bugs, tests that can't be automated, improvements

Playtesting is the highest level of testing, where unit testing was testing the smallest common element of the game, Component testing was testing the interactions between a subset of the components, Playtesting is testing the built game, with all the compoents.
The tests are still aimed at finding bugs, however at this level the only way to test the game is through the VDU(Visual Display Unit) and speakers, which cannot be automated.
It is much easier to identify bugs at this level, since all low level bugs are propagated up to this level, however it is very hard to identify the location of these bugs, which is why the other testing methods are used.
Performance is an aspect included under playtesting, since it can effect the outcome of a siutation within the a real time game.
Playtesters will try different hardware that is considered within the game's minimum system requirements,  to elimiate hardware specific bugs.



\section{Success Measurement}
\label{section:success}

The various aims and goals of the project need to be measured for success individually to ensure that the project is successful overall. The following measures will be used.

\subsection{Stage Gate Model}

The project is organised into separate phases with defined control points, referred to as \emph{stages} and \emph{gates}.\sidenote{This is the stage gate model, see \bibentry{karlstrom2005combining}.} The gates ensure that each phase must be completed to an acceptable level of quality before the project can continue.

The precise measures used at each gate are described in section \ref{section:quality}, but are mostly combinations of acceptance tests and feedback from playtesters.

\subsection{Acceptance Testing}

Acceptance testing is a tried and proved method for making sure project deliverables are of the required quality.\sidenote{See, for example, \bibentry{hsia1994behavior}.} An acceptance test is carried out on behalf of or by the client, and aims to determine if the product has met a certain requirement. A full suite of tests should aim to cover everything that is relevant to quality of a deliverable. The use of acceptance tests in this project is to be fore-mostly at the gates (see above), ensuring that sufficient quality is reached before the next phase of the project can be entered. This both assures quality and avoids over spending in the case the gate cannot be traversed. 

The precise nature of the acceptance tests are to be worked out with the client, and each weekly iteration can be run against them to monitor progress.

\subsection{Playtesting}

% play it, testing gameplay.

The objective of this is to improve the gameplay by poeple playing the game, reviewing it an

The objective of this is to improve the gameplay so it both meets the specification and is fun to play.
Due to the intangability of ``fun gameplay'', a feedback loop is used which is repeated until the changes on each iteration become few or very granulated:
\begin{enumerate}
\item playtester plays the game
\item playtester reviews the game, commenting on aspects they like, don't like and any minor changes to the gamplay believed to improve it
\item all the changes are reviewed, compiling a change list of changes that are agreed to be added
\item the changes are added to the game
\item repeat
\end{enumerate}

