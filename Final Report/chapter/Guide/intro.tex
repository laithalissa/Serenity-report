% Introduction explaining concept of guide
% This is probably the place to reference the phrase "Tackling the Awkward Squad"

\newthought{The previous chapter considered the finished game itself}, in this chapter the technical details of the development are considered, primarily in terms of Haskell and FP. The aim of this chapter is to show that, while some of the techniques used in Haskell programming are different from the more traditional approaches, the overall performance of the language is comparable, and even sometimes favourable, to other wider used platforms.

The idea that target code produced from functional languages is slow and performs badly, and also that functional code is either hard to understand or bad from a software development point of view, persists among some practitioners despite various evidence to the contrary.\cite{roundy2005darcs} It is likely that one of the main reasons for this is unfamiliarity, as mentioned in the main introduction.

It is hoped that the review of techniques used for Project Serenity as reported on in this chapter, can in some way help to reduce this problem. For this reason the findings are given somewhat in the style of a guide: a guide on what to do (and what not to do) when embarking on a large scale Haskell project. Because Project Serenity was a game, the focus is mainly on aspects relevant to game development; but, as mentioned in the introduction, game programming incorporates a great many areas of software development.
