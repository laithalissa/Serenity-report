\section{Rendering the Game}

% old assets
% ----------
% ships made to look very big
% many prototypes
% merging assets with code, dynamic color

% master copy of assets in dropbox under 4y project/laith/design

One very important aspect of the client program is rendering the current game state to a display for the user to interact with. Creating a picture of the game to display requires a number of graphical assets to visually represent the various entities and scenery that make up the game world. For Project Serenity all of these graphics were created from scratch using Photoshop and Illustrator. Figure~\ref{fig:assets} shows a selection of some of the assets that were created for different entities.

% TODO INSERT FIGURE OF PLANETS AND SHIPS
% full page on planets, full ships

% TODO Chat about the actual rendering process. Thoughts:
%
% * World coordinates: all entities are located in world coordinates and a view of the world is drawn in this way first
% * Viewport: the display is scaled and translated by the current settings stored by the client
% * Along with the actual world there is some extra GUI that helps the player understand the current state of the game:
%      - Minimap
%      - Currently selected entity

The client uses these assets to draw the sector and place each player's ships in the correct locations. This is done by placing the assets using world coordinates in the 2D space defined by the dimensions of the sector.

% TODO Talk about parallax because it's interesting? (Innovative?)

Another interesting piece of the rendering puzzle was to choose a colour as a unique visual representation for each player. This would be used to help players differentiate their ships from the enemy and to identify the planets that they own. A clever algorithm was used that would take a numerical player identifier and return a unique colour:\cite{ankerl2009}

\begin{equation*}
	colour(i) = HSV(i \times \phi^{-1} \mod 1, 1.0, 0.8)
\end{equation*}
\noindent
This function uses the HSV colour space using a fixed saturation and value, but modifying the hue to create equally bright and vibrant colours. The player's identifier, $i$, is used to step into the possible values for hue by multiplying it by the reciprocal of the golden ratio, $\phi$, modulo one. Due to the equidistribution theorem\sidenote{The equidistribution theorem states that the sequence $a, 2a, 3a, \ldots \mod 1$ is uniformly distributed when $a$ is an irrational number} this creates a sequence of colours that are evenly spread across the colour space. Using this method to generate team colours created visually pleasing colours that are suitably distinct from each other to be used for recognition.
