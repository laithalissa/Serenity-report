
\section{Server-Client Sychronization}

\begin{comment}

this section talks about the decision for server client architecture.
server-client allows master copy to exist on server.

- master copy of world on server
- clients perform no logic, only apply updates to

we need to work out how the computers talked to each other very early on
2 considerations: peer 2 peer, client-server
  - advantages of peer 2 peer
  - advantages of peer server-client
  - why server-client chosen 
  - issues experienced with server-client
    - jumpy ship movements

\end{comment}

The game was designed around being multiplayer, so our first goal was to decide how they would talk to each other.
Since their was an arbitrary number of players (greater than 1), We needed a system that would scale, but would also be robust.
The game was expecting all players to be on the same LAN (Local Area Network), hence a high bandwidth was available for the communications.


% lockstep
%   - each client has the master copy of the world
%   - when a client does something, it tells all others
%   - problems:
%     - desycnhronization
%     - currupted client fucks up everyone

% simple server client
%   - it's what we've implemented
%   - no simulation on clients
%   - 1 server that has master copy
%   - server does all processing and sends updates to clients
%   - clients just render the world and send commands to server

% server client with simulation
%   - similar to simple server client but client does 'some' simulation. They in no way can effect the world.
%   - for instance if client knows ships destination and current location it can render the ship graciously moving instead of jumping on every update.

