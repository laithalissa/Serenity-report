\section{Game Model}

% what this section talks about
%   - graph of objects and relations
%   - what each model object represents and is
%   - difficulties developing the model
%   - model: sector, game, entity, ship, plan, goal, ship action, ship class, ship configuration, weapon, system, weapon slot, system slot, fleet, Formation, client state
% what is a game model
% early on
% 

This section describe the model that makes the game.
The model describes the different concepts in the game, for instance a weapon, a ship, a formation, etc.
It also describes the relationships between them, for instance a ship has many weapons.
Initially, the game model was simple, but as the design stage progressed, the game model gained complexity.

% sector
The game's terrain is called a sector. The sector has the following fields:
\begin{margintable}
    \begin{tabular}{p{3em} p{9em}}
    \toprule
    \emph{fields} & \emph{description} \\
    \midrule

    name & name of sector \\
    size & size of game \\
    spawn points & the points on the sector where player's fleets are spawned \\
    planets & all the planets in the sector \\
    space lanes & the spaces lanes connecting all the planes \\

    \bottomrule
    \end{tabular}
    	\vspace{1em}
	\caption{sector layout}
	\label{tab:model:sectorFields}
\end{margintable}

A sector is basically a game map.
It specifies the layout of the map, it contains fields that are described in Table \ref{tab:model:sectorFields}.

When the player starts the game, they are able to choose which sector they want to play by its name. 
The spawn points on the sector are typically as far apart as possible. 
This means that bigger sector will typically have longer playing periods relative to smaller sectors.

% planet
Planets in the sector have a variety of properties that make the planet unique.
Table \ref{tab:model:planetFields} gives a breakdown of the planet's properties.

\begin{margintable}
    \begin{tabular}{p{3em} p{9em}}
    \toprule
    \emph{Fields} & \emph{Description} \\
    \midrule
    name & name of planet, used for identifying planets \\
    ecotype & type of planet, eg star, ocean, metal, etc \\
    location & location of planet within the sector \\
    resources & the quantity of resources generated by this planet if captured\\
    \bottomrule
    \end{tabular}
    	\vspace{1em}
	\caption{sector layout}
	\label{tab:model:planetFields}
\end{margintable}

The planet's name is a colloquial name used by players to reference a specify planet.
There a variety of planet ecotypes, some examples shown in Figure \ref{fig:model:starPlanet}, \ref{fig:model:oceanPlanet}, and \ref{fig:model:metalPlanet}.
The ecotype of a planet effects its appearance as well as its distribution of resources.
Some ecotypes will typically have a greater proportion of say metal relative to fuel and anti-matter, whilst another ecotype may favour fuel.

\begin{marginfigure}
	\includegraphics{res/planets/star.png}
	\caption{ecotype: star}
	\label{fig:model:starPlanet}
\end{marginfigure}

\begin{marginfigure}
	\includegraphics{res/planets/OceanPlanet.png}
	\caption{ecotype: ocean}
	\label{fig:model:oceanPlanet}
\end{marginfigure}

\begin{marginfigure}
	\includegraphics{res/planets/metal-planet.png}
	\caption{ecotype: metal}
	\label{fig:model:metalPlanet}
\end{marginfigure}


% ship




% order/plan/goal


% server model

% client model

% 