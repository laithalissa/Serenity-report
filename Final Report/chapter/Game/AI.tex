\section{Artificial Intelligence: Orders, Goals, and Plans}

A sophisticated artificial intelligence (AI) system was highlighted as one of the major
functional requirements for the project. The plan was to use the AI to take high
level orders from the player and convert them into a detailed plan for individual
ships to execute. In this way the player would direct the overall strategy of their
fleet, but without actually captaining the individual ships. This would reduce the
amount of micromanagement a player has to undertake and increase the sense of realism
--- the admiral of a real fleet is unable to directly control individual ships under
their command. This plan for a high level AI was seen as one of the larger challenges
that the project would face.

Artificial intelligence can often be a make-or-break factor in determining the success of
a game.\citepage{rabin2002}{page 3} Without a convincing intelligence system, a game can
quickly become infuriating to play. This is because a human player expects any computer
controlled components to behave sensibly. In some cases well known algorithms exist that
enable `intelligent' behaviour to be implemented relatively easily, for example the use
of the A* search algorithm for pathfinding. However, higher level intelligence systems
are much more challenging. A system capable of creating and executing quality plans
from abstract orders is going to be one of the hardest components to implement.

As well as providing an entertaining experience an AI system must also be efficient.
There cannot be large delays between the user giving an order and it being carried
out. Any planning algorithms have to run quickly otherwise the lag in feedback will
detract from the realism of the game. An inefficient AI system could also stop the game
from running smoothly --- which is of great importance for a real-time strategy game.
This would lead to a poor user experience causing people to stop playing the game.
For these reasons, careful thought was put into designing a system that could fulfill
the important requirements for a successful AI system.

The AI framework that was implemented in Project Serenity is based on a planning
hierarchy. At highest level are orders, such as move to a given location or capture
the specified planet. These orders are under the control of the player, but it is
the job of the AI system to plan a series of actions to acheive them. The first step
is to convert the order into a goal. There is mostly a one-to-one mapping between
orders and goals, for example "OrderMove" maps to "GoalBeAt", however this layer of
abstraction allows the AI to choose unrelated goals if the current order is impossible
or would result in the needless loss of a ship. The goal is then decomposed into a
plan which is a series of actions that are performed to complete the current goal
(and hopefully fulfill the current order too).

The current planning cycle for a ship is as follows:

\vspace{-0.5em}
\begin{listing}{list:planning}{Planning cycle pseudocode}{Pseudocode for the AI framework's planning cycle}{}
\end{listing}\vspace{-1.5em}

\begin{tabbing}
{\bf if} the current plan is empty {\bf then} \\
\quad{\bf if} the current order is complete {\bf then} \\
\quad\quad Reset the ship's order to "OrderNone" \\
\quad\bf else \\
\quad\quad $g \leftarrow$ create goal from the current order \\
\quad\quad $p \leftarrow$ create a plan from the new goal, $g$ \\
\quad\quad Update the ship's goal and plan with $g$ and $p$ \\
\quad\bf fi \\
\bf else \\
\quad{\bf if} the first action in the current plan is complete {\bf then} \\
\quad\quad Remove the action from the head of the ship's plan \\
\quad\bf else \\
\quad\quad Perform the action at the head of the ship's plan \\
\quad\bf fi \\
\bf fi
\end{tabbing}
\noindent
This planning cycle runs every server tick so that all ships in the game can
continuously formulate and update their goals and plans.

An improvement to the planning cycle would be to add in periodic replanning steps.
This would allow a ship to change its plan if the world around it changes, for example
to disregard an order to capture a planet if it is guarded by multiple enemy ships
and there is no friendly support in the area. In the current implementation this is
done on a bit of an ad-hoc basis instead of being formalised into the planning loop.
For example, the code that implements the action to move to another ship will check
if the target ship is still near to its location when the plan was originally formulated.
If the target ship has moved too far then the current plan is scrapped so that a new
plan can be constructed.
