\section{Resources}

\begin{comment}

what are resources

for each resources:
  - what it's used for
  - how it effects the game

mention what their used for

\end{comment}

The resources represent the currency of the game. The player's access to resources can dramatically effect hit chances of winning a battle.

Each resouces has a unique purpose in the game, requiring the player to change strategy dependending on the type and quantity of resources the player has access to.

The three resources are:
\begin{itemize}
\item Fuel
\item Metal
\item Anti-Matter
\end{itemize}

% describe how their are gained (from planets)
Each planet on the map will have a supply of resources. 
The quantity of resources on each planet will vary, but will relate to the planet's type.
When that planet is captured, it will provide a constant trickle of its resources to the player.
For instance if a planet has the following resources:
\begin{description}
\item[Fuel]: 1
\item[Metal]: 2
\item[Anti-Matter]: 0
\end{description}
And is captured by player 1, then every minute, they will receive 2 metal from the planet, and 1 fuel. 
The planet will continue to supply the player with resources until the game ends or another player captures the planet.

% describe how the ship has access to the resources
Each ship will have its own cache of resources on board.
When the ship is near one of it's planets, It will have access to resupply from the player's stockpile. The ship's resource cache will automaticaly refill when near the player's planets. 
Larger class ships have bigger resource caches, and therefore can last longer before needing to resupply (subject to resource consuption).

\subsection{Fuel}
% what it's used for
When a ship's shields are less than 100\%, Fuel is used by the ship to replenish the ship's shields.
The rate that the ship's shields are regenerated is dependent on the ship class, not the quantity of the ship's fuel cache.
If the ship runs out of fuel, it will stop regenerating its shields.

% how it effects the game
When a group of ships leaves the safety of its own planets, and starts entering battles, their fuel supplies will eventually become empty unless they return to one of their planets. 
If the group of ships continue to battle without shields they will sustain damage to hull and will be more likely to loose the battle.
When a battle takes place within proximately to the player's planet, that player will have the benefit of a constant resupply of fuel to the ships in battle. For the other player to win that battle they would need to do so before they ran out of fuel.
This gives the advantag to the defending side.


% describe how each resource is used by the ships


		different planets have different quantities of each resource.
		planets are resource generators, every minute they generate a specific amount of each resource.
		each player has a Fuel/AM/Metal value that is added to their global resource values.

		every ship carries their own supply of F/AM/M which is constantly resupplied own the player's planet graph.
		when exploring/attacking the ships will no longer be on the player's plenet graph hence they will start to eat through the ships resource capacity.

		when a ship is on the player's planet graph they start to resupply draining the player's global resources.





Three resources exist within the game: fuel, metal, and anti-matter. These resources are only used by the ships. Fuel maintains a ship’s shields, without a shield any damage will be done to the ship’s hull. Metal is used to slowly repair a ship’s hull after it has been damaged. Anti-matter is a rare resource that is used by the more powerful weapons available.
Planets generate a constant supply of resources, so players can gain extra resources via planetary capture. The resources gen- erated by the planets owned by a player feed into that player’s global stockpile. Individual ships then draw resources from this stockpile.