%Needs work done on it, needs proof reading etc etc, jo is a squid%

\section{Literature Review: Designing a Game for 2013}
\label{sec:litRevDesigningGame}

The most popular games over recent years have been of the first person shooter, hack and slash, FIFA and Fitness/Dance games genres. Unsurprisingly these genres are among those most frequently published by the publishers shown in Tables~\ref{tab:bestSellingGames2011} and \ref{tab:bestSellingGames2012}, implying that they are significantly more profitable than other genres, which indicates that such genres are also the most popular. This makes designing a new game challenging, since there is an immediate bias towards a genre which shows promise for sustained development and sequel games, and has the largest player base. 

%Decide whether to enumerate or to list
\begin{table*}[!ht]
	\begin{tabular}{p{15em} p{13em}}
		\toprule
		\emph{Title} & \emph{Publisher}\\
		\midrule
	Call of Duty: Modern Warfare 3 & Infinity Ward
	\\
	FIFA 12 & Electronic Arts
	\\
	Battlefield 3 & Electronic Arts, Sega
	\\
	Zumba Fitness & Majesco Entertainment
	\\
	The Elder Scrolls V: Skyrim & Bethesda Softwork
	\\
	Just Dance 3 & Ubisoft
	\\
	Assassin's Creed: Revelations & Ubisoft
	\\
	LA Noire & Rockstar Games
	\\
	Saints Row: The Third & THQ, CyberFront, MicroByte
	\\
	Batman: Arkham City & Warner Bros. Interactive Entertainment 
	\\
	\bottomrule
	\end{tabular}
	\caption[Best selling games of 2011]{Best selling games of 2011.}
	\label{tab:bestSellingGames2011}
\end{table*}
%% todo: put reference \cite{bestgames2011}

\begin{table*}[!ht]
	\begin{tabular}{p{15em} p{13em}}
		\toprule
		\emph{Title} & \emph{Publisher}\\
		\midrule
	Call of Duty: Black Ops II & Activision Blizzard
	\\
	FIFA 13 & Electronic Arts
	\\
	Assassin's Creed III & Ubisoft
	\\
    Halo 4 & Microsoft	
    \\
	Hitman Absolution & Square Enix
	\\
	Just Dance 4 & Ubisoft
	\\
	Far Cry 3 & Ubisoft
	\\
	FIFA 12 & Electronic Arts
	\\
	The Elder Scrolls V: Skyrim & Bethesda Softworks
	\\
	Borderlands 2 & 2k Games
	\\
	\bottomrule
	\end{tabular}
	\caption[Best selling games of 2012]{Best selling games of 2012.}
	\label{tab:bestSellingGames2012}
\end{table*}
%% todo: put reference \cite{bestgames2012}

The highest priority when designing a game is to make it fun for the target audience, however, the reception of a game is heavily dependant on the ever-changing  gaming culture, so it's extremely difficult to formalise the art of game development. Consequently academic references are extremely rare, so the project's literature review aims to identify current trends by investigating a range of publications, from leading game designers such as \emph{Valve corporation} \emph{EA Games}, \emph{Ubisoft}, as well as the leading game critics \emph{GameSpot}, \emph{IGN}, and independent game reviewers \emph{Tom Francis}, \emph{Anton Temba}, \emph{Wolfgang Kramer} and \emph{Peter Collier}.\cite[-2em]{makeuseof, ebizmba}

\ 

% Literature review on our game design

\begin{description}

\item[Originality] Francis and Kramer both assert that a successful game should have some original content such as a unique storyline, an unusual physics engine, new weapon concepts, or anything which pulls away from conventional games and adds a new dimension to gaming. If a game has limited originality then it risks becoming dull and uninteresting since it will appear to be a combination of existing game features instead of offering a new concept to the gaming community.\cite{tomfrancis} \cite{wolfgangkramer} 

\item[Adaptive Pacing and Replayability] Temba highlights the need for the gaming experience to be fuelled by lots of active events, where the player can make choices to influence the outcome. Kramer echoes this view by expressing the need for a unique gaming experience developed by carefully architecting the game. \cite{antontemba} \cite{wolfgangkramer} Valve addresses the issue of replayability through ``algorithmically adjusting the game pacing on the fly to maximise ``drama'' (player excitement/intensity)''.\cite{valveAI} and spends a lot of design time investigating a procedurally populated environment. The purpose of these adjustments is to fit the gameplay to the individual player so the game is always challenging and exciting, which can be particularly difficult for cooperative multiplayer games.

Many approachable solutions for the project to create a unique gaming experience include procedurally generated environment, intelligently offering \emph{supplies} or bonus upgrades (temporary invulnerability, invisibility etc) to losing players, to ensure the game is balanced and enjoyable for players of all abilities.

\item[Surprises] Game journal \emph{Gameranx} highlights the need for surprises in both the plot and hidden features, which ``extend the length and replay value of a game, and more often than not give gamers something to talk about''.\cite{gameranx} A typical feature of a well structured game is to have different possible outcomes for every possible decision the player is allowed to make. This is a rather typical pattern found in games such as Unreal Tournament, where a player's weapon arsenal might allow him to fend off foes competing for an objective.
% UT, 

\item[Equal opportunity]
Francis and Wolfgang both consider equal opportunity is vital for multiplayer games, and so it's important that no player has a distinct advantage over any other human player; such imbalances could be caused by advantageous starting positions, unequal starting budgets or  anything which gives one player a better chance of winning. Clearly, different games have different requirements for fulfilling equal opportunity, but one of the most important requirements of a multiplayer game is to give every player an equal chance to win the game from the starting point. \cite{tomfrancis} \cite{wolfgangkramer}

\item[No early elimination]
No player should lose hope of winning early in the game. In the event of poor playing in the start of the game, a player should be able to redeem themselves to regain a chance of winning the game, in this context it's acceptable for the game to \emph{help} a player (e.g. spawning health boosts near the player, or having the AI teams primarily target other players. Valve introduced AI assisted power-ups in \emph{Left 4 Dead 2}, where the game designer specifies available supplies and the game's AI decides when to spawn them based on the player's needs, which reduces the chance of the player being unable to progress due to earlier difficulties. \citepage{valveAI}{slide 75}

\item[Low waiting times]
Francis and Wolfgang agree that a player's interest lies in being involved with a game and don't want to endure period of low activity. This was a common problem in early real-time-strategy games such as Age of Empires, where a player would spend significant time in the ``setting up'' phases. It's important to balance the focus of the game so there's focus on areas of interest without risking the compromising the story or atmosphere. \cite{tomfrancis} \cite{wolfgangkramer}
% todo: check hyphenation of rts

\item[Environmental Art and Sound]
Valve argues that much of the game's atmosphere stems from the player's environment, so atmospheric artwork and sound should be considered an important factor in making the players feel immersed in the game.\citepage{valveSP}{Slide 8}


\end{description}


This project intends to challenge the conventional genre space by aiming to produce a real time strategy game. The RTS genre has given rise to some extremely popular productions in the early day of gaming corporations such as Total Annihilation, Supreme Commander and the Command and Conquer franchise. Early games lacked sophisticated hardware and game engines, so there was a large focus on developing the gameplay experience, which is likely why these games are so iconic. Since time frame and budget of the project is limited by time and resources, it seems that a simple RTS game would be suitable choice of genre for the game provided it does not interfere with the criteria previously outlined in this section.
