\newthought{Functional programming} (FP) has a long history, with its roots in the $\lambda$-calculus of Alonzo Church.\citefix[-1.5em]{church1932} One of the first functional programming languages was Lisp, invented by John McCarthy in 1958, which is still used today, over 50 years later.\citepage{reilly2003}{pages 156--157} Various languages have refined and extended the functional paradigm over the years --- probably the most notable as of now being Haskell, Scala, OCaml, F\#, and Erlang.

Despite the amount of time such languages have been available, use in industry has typically been far less than that of languages such as C, C++, and Java.\citepage[-2em]{odersky2010programming}{page 11} That being said, in recent years there has been increasing use of functional techniques and languages in certain areas. Erlang was designed for the development of highly fault tolerant telecommunication systems.\cite[-1em]{armstrong2007history} OCaml is used extensively by some organisations in the financial sector to create trading algorithms and other similar applications.\citefix[1em]{minsky2011ocaml} Scala is also increasingly popular in various domains, helped in part by its compatibility with the Java Virtual Machine (JVM) and object oriented design. And finally Haskell has gone from strength to strength in recent years, is used in diverse problem domains, and has a very active development community.\sidenote[][-1em]{For example: \bibentry{hawkinscontrolling}. For other instances see the \emph{Commercial Users of Functional Programming} (CUFP) Workshop, \url{cufp.org}, and the \emph{Industrial Haskell Group} (IHG), \url{industry.haskell.org}.}

One of the most often cited reasons against the use of functional programming in some domains is that of performance. This is due in part to mutable data structures generally being easier to represent on machine hardware; and it therefore being harder for functional compilers to convert the code into an efficient representation.\citefix{paulson1996ml} However, it is not a given that any program would run slower if written in a functional language: in some cases lazy-evaluation or compiler optimisations made possible by immutability can mean a program runs faster, plus advanced compiler techniques such as array fusion can lead to programs nearing the efficiency of hand-crafted C. And with modern machines getting ever faster, the domain of problems that require high levels of efficiency is getting smaller.

Performance problems alone cannot account for the fringe position of functional programming. The efficacy of the functional approach has been touted for many years,\sidenote{For example see \bibentry{hughes1989functional}.} yet it is still rare for mainstream projects to make any use of functional languages. A possible reason for this is familiarity: because most large scale and enterprise software projects are written in languages such as Java and C++, knowledge about functional techniques and design patterns among professionals is low. And, of course, this means that new projects are unlikely to take up Haskell or other functional languages; and in absence of direct experience outdated beliefs can remain.

Rather than researching and discussing the theoretical advantages of the functional paradigm, this project instead attempts to demonstrate the value of functional programming by utilising it in a problem domain that should pose a significant challenge, and not normally considered a `good' domain for FP. The chosen domain is the development of a computer game. By developing the software entirely in a functional language and documenting the process thoroughly, it is hoped to provide valuable insight for future projects that could benefit from a functional approach, and also to dispel myths that functional code is slow or hard to maintain.x

This choice was made for several reasons. Game programming brings together a diverse range of computing areas --- for example human interaction in real time, detailed graphics and animation, artificial intelligence / planning, networking, and various other dynamic elements.\sidenote{See \bibentry{crawford1984art}.} For these reasons games are often at the forefront of technological and theoretical advances in computing.\citefix[1em]{haddon1988electronic} However, a game is also a tangible, sizeable piece of software, yet achievable for a four person group over two terms. 

As well as demonstrating FP over a wide range of areas, a game also represents a serious business venture.\cite[1em]{essentialFacts2012} Computer games have been a huge industry for almost as long as personal computers have existed. Demand is high, and a vast amount of games are being continually developed --- from triple-A ventures and large companies, to independent (``indie'') companies and hobby groups.

The game developed as part of this project is certainly not the first game ever to be developed in a functional language, nor is it likely to be the last. But by fully documenting and explaining the process, it is hoped to provide a valuable insight into the effectiveness of FP for both game development and other professional computing applications.

