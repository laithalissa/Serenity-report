\section{On the Choice of Language for the Project}

The following factors were used to judge suitability between languages for use in the project.

\begin{description}
	\item[Strength of Type System] It was desired to use a strong, advanced type system to enable the full benefits of advanced FP techniques.
	\item[Purity] Many of the advantages of FP come only when true separation between pure functional code, and impure `effectful' code, is available.\cite{hudak1989conception}
	\item[Concision] Concise but readable code syntax was preferred to verbose or obscure.\sidenote[][1em]{\bibentry{brooks1995},
also \bibentry{taliaferro1971modularity}, and
\bibentry{wolverton1974cost}.}
	\item[Speed] Due to the domain, a reasonable degree of performance was required of the target code.
	\item[Testing] Good testing support was desirable both from a software development point of view, and the ability to show test-driven techniques in an FP context.\sidenote[][1em]{See \bibentry{beck1999}.}
	\item[Community and Library Support] Good availability of libraries and a thriving community all vastly aid in development.
	\item[Familiarity] Languages the team already had some experience of would allow for more effort to be spent on the actual project than learning a completely new language, but also any language that was \emph{too} familiar might make some results less applicable.
\end{description}

\subsection{Lisp / Scheme / Clojure}
\subsection{Scala}
\subsection{Other ML variants (SML, OCaml, F\#, etc)}
\subsection{Haskell}
\subsection{Erlang}