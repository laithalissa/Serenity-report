\section{Presentation of Results and the Overall Structure of this Document}

At the conception of this project it was known only that some kind of game was to be developed with a functional language. Haskell was identified as the language early on as discussed above. The next phase of the project was to conduct research into existing work, build some pilot games to ensure that the desired outcomes were possible, and to combine this work into a full specification for the future work. This was undertaken mostly during the summer before the first university term (in 2012), and in the first few weeks. The specification was delivered in Week 4, on the 25th October. This work is presented in Chapter~\ref{ch:rd}.

The next, and main stage of the project was the actual development of Project Serenity. This work is examined from two different angles in this report. Firstly, the end software at the time of publication is considered in Chapter~\ref{ch:game} from the point of view of a software product, and how the process of the development proceeded. Also considered are future directions work into Project Serenity might take. 

Separately from this is a detailed report on insights into the use of FP for game development: best practices, areas that FP excels in, and areas in which problems were encountered, presented in the form of a `guide' the functional game programming. This forms the contents of Chapter~\ref{ch:guide}.

Chapter~\ref{ch:pm} is a full evaluation of the project management techniques used throughout, and the success of the project itself. Business value and other professional issues are also addressed.

Finally, the full results are summarised and their implications are considered in Chapter~\ref{ch:conclusions}. A introduction to some basic functional design patterns is included as an appendix.