\section{Development Model}
\label{sec:devmodel}

An agile approach to software development was chosen for this project. This methodology was
chosen because of its focus on rapid development and handling change. When working in a small
team, a heavy weight plan-driven development approach can dominate the actual process of development
due to the overhead. Sommerville states that using such an approach means that ``more time is
spent on how the system should be developed than on program development and testing.''\citepage{sommerville2011}{page 58}
In contrast, an agile approach is designed to deliver working software quickly so that changes
can be suggested and implemented in future iterations.

The agile method that was used was that of extreme programming developed by Kent Beck.\cite{beck1999}\cite{beck2000}
Extreme programming is an `extreme' approach to iterative development. Small releases are made
as quickly as possible. These releases are then evaluated and iterated on until a final release
that meets all requirements is delivered. Instead of planning and designing for the far future,
extreme programming advocates doing both of these activities --- bits and pieces at a time ---
throughout the entire project development lifecycle.

% Bad at frequent, formal release cycle

One core practice at the centre of extreme programming is testing and test driven development.
Test driven development, as described previously in section~\ref{sec:testing}, involves
developing test cases before coding the actual implementation. The benefits of test driven
development are a system that is thoroughly tested, reduced ambiguities in specification before
implementation begins, and avoidance of `test-lag'. However, Sommerville notes a few problems
that can be encountered when using test driven development:

\begin{enumerate}
\item Programmers prefer programming to testing. It is very tempting for a programmer to write
      incomplete tests, or skip test writing altogether, before moving onto the more rewarding
      task of implementation.

\item In some cases tests can be difficult to write. For example, testing user interfaces and
      display logic.

\item It is hard to judge the completeness of a test suite. There may be a large number of tests,
      but do they actually cover all of the code and all possible program execution.
\end{enumerate}

Although the Serenity project does include a test suite the team failed to stick to the test
driven development tenet of extreme programming. This was mainly caused by the first issue
pointed out by Sommerville. The team preferred to go straight into implementing a feature or
enhancement and then, maybe, write tests after the fact.

Another important practice in extreme programming is that of pair programming. Two developers
work in tandem at the same computer. One programmer, the driver, actively writes the implementation
of the program. The other, the observer, continously reviews each line of code as it is typed 
and thinks about the direction of the work.\cite{williams2001} There are a couple of major advantages
to pair programming:

\begin{enumerate}
\item It is an informal code review process that can be very effective at discovering errors
      as the code is written.

\item It promotes collective ownership and responsibility for the component being worked on.
      Code is not `owned' by an individual who may dislike others working in the same area or
      be demotivated by critism during code reviews, similarly one individual is not held responsible
      for any problems.
\end{enumerate}

Studies have shown that pair programming may have little effect on overall productivity, but
creates a substantial reduction in errors in the code.\cite{cockburn2000} This is prescribed
to the continuous code review that occurs, and a decrease in false starts and redoing work.

Pair programming was an effective technique that was used throughout the development of
Serenity. The experience lead to the conclusion that pair programming is a good method for
development for the advantages mentioned above as well as the following reasons:

\begin{description}
\item[Problem solving] When a problem is encountered two people are able to discuss it together
which often helps either or both of them coming up with a solution quicker than they would
individually.

\item[Learning] Knowledge is constantly exchanged within the pair. So after a component
has finished and the pair move on to other work they have both become more effective programmers
in some way.

\item[Team building] Working together lead to better communication and enhanced teamwork.
This made the project team a more effective work group.
\end{description}

However, due to such a small team it was felt that it would be impossible to work on all
tasks in pairs and still finish the work within the time constraints. Therefore, some work
was done individually, but still always trying to work in close proximity to enable teamwork
where necessary.
