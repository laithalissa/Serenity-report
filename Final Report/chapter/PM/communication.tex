\section{Communication}

The various stakeholders that were identified during the specification stage are shown in Table~\ref{tab:stakeholders}.
The relationship between the stakeholder and the project is shown, along with a rough
estimate of their power and interest.\sidenote[][-2em]{See \bibentry{mendelow1991stakeholder}}
This grid formed a reference for making sure that all interested parties were communicated
with appropriately throughout the duration of the project.

This communication plan was developed early during the project specification to enable communication to be effectively prioritised and maintained. Here the efficacy of the communication plan is reviewed.

\subsection{Supervisor Meetings}
Prior to starting the project it was thought that regular communication with the project
supervisor was likely to be a critical factor in success of the project. Therefore the
possibility of weekly meetings with at least one member of the group if not more were
planned. During the project the feedback from these meetings from the supervisor was very
helpful in ensuring that the project was kept on track and that the goals were always
kept in mind. Although the meetings were not as frequent as planned there were several
per term. Also, the team made their calendar available to the supervisor so that she
could drop by during the planned coding sessions that occurred twice weekly.

\begin{table*}
	\footnotesize
	\renewcommand{\arraystretch}{1.5}
	\begin{tabular}{p{9em} p{5em} p{3em} p{3em} p{9em} p{9em} p{9em}}
		\toprule
		\emph{Stakeholder} & \emph{Relationship} & \emph{Power} & \emph{Interest} & \emph{Requirements} & \emph{Measurements} & \emph{Communication Strategy} \\
		\midrule
		
		Project Team & Internal & High & High & 
		Good working environment, creative input. & 
		Meeting project spec, good grades! & 
		Various, detailed elsewhere. \\
		
		Supervisor --- Sara Kalvala & Internal & High & High & 
		Good communication. & 
		Adherence to spec, good PM, high quality write-up. & 
		Weekly meetings. \\
		
		Client --- Matt Leeke & Core \mbox{External} & High & High & 
		Good communication, creative input, hard work & 
		Strength of software, strength of report & 
		Weekly meetings. \\
		
		Second Assessor & Core \mbox{External} & High & Low & 
		None & 
		Marking scheme & 
		Deliverables only. \\
		
		Projects Organiser --- Steve Matthews & External & High & Low & 
		Cooperation when required. & 
		Deliverables on time. & 
		Email or meeting if required. \\
		
		Playtesters & External & Low & High & 
		Able to report issues / feature requests. & 
		Strength of game, input considered. & 
		Email. \\
		
		Other future users & Rest of World & Low & High & 
		Game works and is reliable. & 
		Strength of game, re-playability. & 
		Website, forums, blog. \\
		
		The Haskell and FP Communities & Rest of World & Low & High & 
		None & 
		Interest in  / strength of results and tools released. & 
		Online as above, and via the final report. \\
		\bottomrule
	\end{tabular}
	\vspace{1.5em}
	\caption[Stakeholders for the project][1em]{Stakeholders for the project.}
	\label{tab:stakeholders}
\end{table*}

\subsection{Client Meetings}

The client is clearly vital to the success of the project, and so it was planned for there to be weekly feature releases to the client along with meetings to discuss the changes. It was thought that continual
feedback on each release would allow for early identification of any problems in the game.
In reality these meetings were not actually this
frequent, due to time constraints on the client as much as on the project team. However, feedback from the client was always positive and no
large changes needed to be made to the project requirements. 
The approach taken by the client was commendable, providing a valuable source of feedback and helping to maintain focus, without attempting to micromanage or control the nature or creative freedom of the development.
Indeed, the project team could have made more effort in keeping up communication with the client and taking advantage of his input.

\subsection{Projects Organiser and Second Assessor}

The projects organiser could exert a strong influence over the project if they wished,
but as there are many projects and it would be inappropriate for them to demonstrate
partiality, therefore extended levels of communication were not necessary. Brief updates
pertaining to deliverables were all that were required. However, the team was aware that
if the project organiser initiated communication then they should be made a high priority.

It was felt that communication with the second accessor was, for the most part, not
appropriate, excepting when delivering this final report and various presentations. This approach seems to have been correct.

\subsection{Playtesters and End Users}

End users are clearly important to the goals of the project, but they had little
interest or influence during the early stages. Therefore, infrequent updates via email
and a mailing list for any events that are organised were sufficient communication with
this group.

There was not time in the end schedule for nearly as much playtesting as was desired, and so this communication plan has remained largely untested. However, it is clear that good communication would be required to maintain interest and give feedback on bug reports etc.

\subsection{The Haskell and Functional Programming Communities}

The overall end goal of this project is not just the game that was developed, but an examination
of Haskell and Functional Programming as a game development environment. However, the Haskell community at
large is unlikely to have much interest in the project during its development. Communication
back to the community should therefore be largely via this final report, as well as the
methods for end users above.
