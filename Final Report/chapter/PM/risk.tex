\section{Risk Management}
\label{section:risk}

A proactive approach to risk management was taken. This technique was chosen in order
to maximise the probability of avoiding risks instead of having to move into `fire-fighting mode'
if something went wrong.\citepage{pressman2010}{page 745}

As part of this proactive risk management strategy, a number of potential risks were identified.
These risks are shown in table \ref{tab:risks} along with their estimated probabilities of occurring
and impact if they were to occur.

\begin{table*}
	\small
	\begin{tabular}{l p{\textwidth / 2} l l}
		\toprule
		\emph{Risk} & \emph{Description} & \emph{Probability} & \emph{Impact} \\
		\midrule
		Length underestimate & The time required to develop the software is underestimated & Medium & High \\
		Team member illness & One or more team members unable to work due to illness & Medium & High \\
		Hardware failure & Damage to critical hardware causing loss of data & Medium & Medium \\
		Size underestimate & The size of the deliverable has been underestimated & Medium & Medium \\
		Requirements change & Large number of changes to requirements during development & Low & Medium \\
		Ambiguous requirements & Requirements are not fully understood or misinterpreted leading to
			loss of development time as the specification is recreated & Low & Medium \\
		\bottomrule
	\end{tabular}
	\vspace{1.5em}
	\caption{Risk identification and analysis.}
	\label{tab:risks}
\end{table*}

With the risks identified, and their likelihood and consequences estimated it is necessary
to draw up plans to mitigate their effects. There are three types of management strategies 
for individual risks: avoidance strategies to reduce the probability of the risk occurring;
minimisation strategies to reduce the impact of the risk; and contingency plans to deal with
the risk if it does arise.\citepage{sommerville2011}{page 601} It is best to avoid the risk,
but if this is not possible then minimisation of the effects and, finally, contingency plans
should reduce the overall impact of a risk on the project. The mitigation and management
strategies for each risk previously identified are listed in table \ref{tab:rmm}.

\begin{table*}
	\small
	\begin{tabular}{l p{37em}}
		\toprule
		\emph{Risk} & \emph{Mitigation / Management} \\
		\midrule
		Length underestimate & Detailed work breakdown with weekly releases to ensure that
			schedule slippage can be caught early \\
		Team member illness & Well documented code (enforced by the software librarian) so
			that other members can quickly start work on less familiar sections of
			the codebase \\
		Hardware failure & Backups and distributed source control, see \emph{Tools and Techniques} \\ % XXX
		Size underestimate & Detailed work breakdown structure \\
		Requirements change & Thorough change management system, see \emph{Change Management} \\ % XXX
		Ambiguous requirements & Thorough planning phase \\
		\bottomrule
	\end{tabular}
	\vspace{1.5em}
	\caption{Risk mitigation and management.}
	\label{tab:rmm}
\end{table*}

The final stage of the risk management process is monitoring. Throughout the duration of
the project each identified risk was reassessed for changes to its probability and
impact. This allowed the mitigation and management strategies to be revisited to ensure that
they were as effective as possible.

The two previously identified risks that actually occurred were team member illness and length underestimate.
On a couple of occasions a team member was ill and unable to attend group work sessions or work to their full
capacity. Fortunately, the team was able to reduce the impact of this by ensuring that the work each individual
was performing was not hindered by an absence. This was done by allocating work tasks to be as separate as
possible to allow more parallel development to occur. Also, an illness was never so severe as to stop a team
member from working for longer than a day or two. However, the issue of length underestimation was more serious.

% Length underestimation
%
% 1. More time required for network and GUI than hoped
% 2. (1) slowed the development of the ideal game
% 3. (1) was very informative for our goal of investigation of Haskell for game development

The proactive approach to risk management was a good choice. By reviewing the potential
risks before starting the development phase of the project it was much easier to avoid risks that could
have had disastrous consequences for the project. For example, by implementing a thorough backup strategy
prior to any data loss actually taking place it was ensured that no work would have been lost if a hardware
failure had occurred. Continuously monitoring and reassessing these risks was also helpful in preventing
any risks becoming more probable or having a greater impact.
