\section{Risk Analysis}
\label{section:risk}

\begin{enumerate}
\item
conflicting with intellectual property

\item
Under Scheduling:
Software projects are typically unique by nature.
schedules can be created using existing projects as a baseline, the greater the similiarity between two projects, the similar the schedules can be.
This project is highly unique, meaning the estimates for each activity will not be based on any previous evidence.
By working in short increments, we will be able to constantly evaluate our progress against the schedule, reachusting where necessary.

\item
Requirement Inflation:
As development progress, new features that weren't identified in the initial specification will be required.
This will add additional activities to the schedule which could cause iterations to be late.

\item
Project Members specialising in certain areas:
As implementation progresses, project members will work on areas of the project they have experience in, interest in, or had previously written.
However if any member becomes ill/unable to continue work, the remaining team will be delayed by having to learn how those areas of the project worked.

\item
Interpretation of Specification:
everything that is not specificed in the specification is ambigues and available for interpretation by all members of the team.
This ambiguities are likely not to be found until integration begins, which could result in certain sections of the project very different from the intended functionality.

\item
Specification Conflictions:
when coding begins, it becomes apparent that the specification is either too abstract or contains conflictions.

\item
Working efficency proportional to the length of the deadline:
the longer the deadline, the less efficent the team memebers will be, due to lack of pressure.

\end{enumerate}

The remainder of this section describes with some the foreseeable challenges that have been
identified. These are less quantifiable than the risks detailed in the table above.

\subsection{Time constraints}

Game development projects are famous for scheduling issues that threaten to delay the
release of a product. Developers often find themselves facing ``crunch time", a period
of extreme work overload, in an effort to deliver a game on time.\cite[-1em]{groen2011}
A survey of problems encountered in game development performed by Petrillo et al. found
that two of the most common issues are missing deadlines and crunch time that results 
from this.\cite[1em]{petrillo2009} Although delays are a challenge common to all projects,
the survey found that the need for multiple disciplines working together (programming,
graphic design and music composition for example) to create a quality game causes
deadline problems to occur even more frequently. These common problems have their roots
in the time constraints imposed on a particular project.

This project has approximately twenty five weeks in which to develop a fully functioning
game that meets the requirements specified previously. This is a relatively short amount
of time in which to deliver a complex game. By adhering to the project management
and software development techniques laid out previously it is hoped that the project
can be kept on schedule and the final deliverable be released on time and to specification.

In his essays on software development, Frederick Brooks argues that the complex
communication structures in a team is a major cause of delays to software projects.\cite{brooks1995}
Fortunately, this project is run by a small team of four and so should find that
communication overhead is less of a problem. The problem of bringing any new team
members up to speed can also be ignored since this is a static team.
However, the short time frame available for completing the project is still a
major challenge to be overcome.

\subsection{Writing a successful AI}

Artificial intelligence can often be a make or break factor in determining the success of
a game.\citepage{rabin2002}{page 3} Without a convincing intelligence system, a game can
quickly become infuriating to play. This is because a human player expects any computer
controlled components to behave sensibly. In some cases well known algorithms exist that
enable `intelligent' behaviour to be implemented relatively easily, for example the use
of the A* search algorithm for pathfinding. However, higher level intelligence systems
are much more challenging. A system capable of creating and executing quality plans
from abstract orders is going to be one of the hardest components to implement.

As well as providing an entertaining experience an AI system must also be efficient.
There cannot be large delays between the user giving an order and it being carried
out. Any planning algorithms have to run quickly otherwise the lag in feedback will
detract from the realism of the game. An inefficient AI system could also stop the game
from running smoothly --- which is of great importance for a real-time strategy game.
This would lead to a poor user experience causing people to stop playing the game.

\subsection{Efficiency problems}

\subsection{Minimal graphics libraries available}

Some investigation into the Haskell graphics libraries available has already been undertaken.
The Gloss package has been identified as a suitable candidate because it exposes a clean
functional API and hides away the details of OpenGL. Unfortunately it is a relatively simple
library and does not provide some required features such as windowing and clipping.

\section{Risk Management}

Unexpected delays need to be anticipated and dealt with promptly, although risks present themselves in different forms, the ultimate cost of delays is the loss of time which can prove disasterous to precariously scheduled projects. Our project schedule anticipates delays by including feed buffers between components of the project for dealing with small delays without suffering the overhead of returning to complete the task at a later stage. In addition to feed buffers our schedule has allocated a project buffer, a reserve of time allocated to optional components (such as extended gameplay modes) which can be reallocated to deal with major delays.
