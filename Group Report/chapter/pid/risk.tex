\section{Risk Analysis}
\label{section:risk}

\begin{enumerate}
\item
conflicting with intellectual property

\item
Under Scheduling:
Software projects are typically unique by nature.
schedules can be created using existing projects as a baseline, the greater the similiarity between two projects, the similar the schedules can be.
This project is highly unique, meaning the estimates for each activity will not be based on any previous evidence.
By working in short increments, we will be able to constantly evaluate our progress against the schedule, reachusting where necessary.

\item
Requirement Inflation:
As development progress, new features that weren't identified in the initial specification will be required.
This will add additional activities to the schedule which could cause iterations to be late.

\item
Project Members specialising in certain areas:
As implementation progresses, project members will work on areas of the project they have experience in, interest in, or had previously written.
However if any member becomes ill/unable to continue work, the remaining team will be delayed by having to learn how those areas of the project worked.

\item
Interpretation of Specification:
everything that is not specificed in the specification is ambigues and available for interpretation by all members of the team.
This ambiguities are likely not to be found until integration begins, which could result in certain sections of the project very different from the intended functionality.

\item
Specification Conflictions:
when coding begins, it becomes apparent that the specification is either too abstract or contains conflictions.

\item
Working efficency proportional to the length of the deadline:
the longer the deadline, the less efficent the team memebers will be, due to lack of pressure.

\end{enumerate}

\section{Risk Management}
Unexpected delays need to be anticipated and dealt with promptly, although risks present themselves in different forms, the ultimate cost of delays is the loss of time which can prove disasterous to precariously scheduled projects. Our project schedule anticipates delays by including feed buffers between components of the project for dealing with small delays without suffering the overhead of returning to complete the task at a later stage. In addition to feed buffers our schedule has allocated a project buffer, a reserve of time allocated to optional components (such as extended gameplay modes) which can be reallocated to deal with major delays.