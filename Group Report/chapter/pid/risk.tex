\section{Risk Management}
\label{section:risk}

A proactive approach to risk management has been taken. This is to maximise the probability
of avoiding risks instead of having to move into `fire-fighting mode' if something goes
wrong.\citepage{pressman2010}{page 745}

As part of this proactive risk management strategy, a number of potential risks
have been identified. These risks are shown in table \ref{tab:risks} along with their
estimated probabilities of occurring and impact if they do occur.

\begin{table*}
	\small
	\begin{tabular}{l p{\textwidth / 2} l l}
		\toprule
		\emph{Risk} & \emph{Description} & \emph{Probability} & \emph{Impact} \\
		\midrule
		Length underestimate & The time required to develop the software is underestimated & Medium & High \\
		Team member illness & One or more team members unable to work due to illness & Medium & High \\
		Hardware failure & Damage to critical hardware causing loss of data & Medium & Medium \\
		Size underestimate & The size of the deliverable has been underestimated & Medium & Medium \\
		Requirements change & Large number of changes to requirements during development & Low & Medium \\
		Ambiguous requirements & Requirements are not fully understood or misinterpreted leading to
			loss of development time as the specification is recreated & Low & Medium \\
		\bottomrule
	\end{tabular}
	\vspace{1.5em}
	\caption{Risk identification and analysis}
	\label{tab:risks}
\end{table*}

With the risks identified, and their likelihood and consequences estimated it is necessary
to draw up plans to mitigate their effects. There are three types of management strategies 
for individual risks: avoidance strategies to reduce the probability of the risk occurring;
minimisation strategies to reduce the impact of the risk; and contingency plans to deal with
the risk if it does arise.\citepage{sommerville2011}{page 601} It is best to avoid the risk,
but if this is not possible then minimisation of the effects and, finally, contingency plans
should reduce the overall impact of a risk on the project. The mitigation and management
strategies for each risk previously identified are listed in table \ref{tab:rmm}.

\begin{table*}
	\small
	\begin{tabular}{l p{37em}}
		\toprule
		\emph{Risk} & \emph{Mitigation / Management} \\
		\midrule
		Length underestimate & Detailed work breakdown with weekly releases to ensure that
			schedule slippage can be caught early \\
		Team member illness & Well documented code (enforced by the software librarian) so
			that other members can quickly start work on less familiar sections of
			the codebase \\
		Hardware failure & Backups and distributed source control, see \emph{Tools and Techniques} \\
		Size underestimate & LOL! \\
		Requirements change & Thorough change management system, see \emph{Change Management} \\
		Ambiguous requirements & ROFL! \\
		\bottomrule
	\end{tabular}
	\vspace{1.5em}
	\caption{Risk mitigation and management}
	\label{tab:rmm}
\end{table*}

The final stage of the risk management process is monitoring. Throughout the duration of
the project each identified risk will be reassessed for changes to its probability and
impact. This allows mitigation and management strategies to be revisited to ensure that
they are as effective as possible.
