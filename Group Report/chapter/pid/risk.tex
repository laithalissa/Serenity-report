\section{Risk Management}
\label{section:risk}

A proactive approach to risk management has been taken. This is to maximise the probability
of avoiding risks instead of having to move into `fire-fighting mode' if something goes
wrong.\citepage{pressman2010}{page 745}

As part of this proactive risk management strategy, a number of potential risks
have been identified. These risks are shown in table \ref{tab:risks} along with their
estimated probabilities of occurring and impact if they do occur.

\begin{table*}
	\small
	\begin{tabular}{l p{\textwidth / 2} l l}
		\toprule
		\emph{Risk} & \emph{Description} & \emph{Probability} & \emph{Impact} \\
		\midrule
		Length underestimate & The time required to develop the software is underestimated & Medium & High \\
		Team member illness & One or more team members unable to work due to illness & Medium & High \\
		Hardware failure & Damage to critical hardware causing loss of data & Medium & Medium \\
		Size underestimate & The size of the deliverable has been underestimated & Medium & Medium \\
		Requirements change & Large number of changes to requirements during development & Low & Medium \\
		Ambiguous requirements & Requirements are not fully understood or misinterpreted leading to
			loss of development time as the specification is recreated & Low & Medium \\
		\bottomrule
	\end{tabular}
	\vspace{1.5em}
	\caption{Risk identification and analysis}
	\label{tab:risks}
\end{table*}

With the risks identified, and their likelihood and consequences estimated it is necessary
to draw up plans to mitigate their effects. There are three types of management strategies 
for individual risks: avoidance strategies to reduce the probability of the risk occurring;
minimisation strategies to reduce the impact of the risk; and contingency plans to deal with
the risk if it does arise.\citepage{sommerville2011}{page 601} It is best to avoid the risk,
but if this is not possible then minimisation of the effects and, finally, contingency plans
should reduce the overall impact of a risk on the project. The mitigation and management
strategies for each risk previously identified are listed in table \ref{tab:rmm}.

\begin{table*}
	\small
	\begin{tabular}{l p{37em}}
		\toprule
		\emph{Risk} & \emph{Mitigation / Management} \\
		\midrule
		Length underestimate & Detailed work breakdown with weekly releases to ensure that
			schedule slippage can be caught early \\
		Team member illness & Well documented code (enforced by the software librarian) so
			that other members can quickly start work on less familiar sections of
			the codebase \\
		Hardware failure & Backups and distributed source control, see \emph{Tools and Techniques} \\
		Size underestimate & LOL! \\
		Requirements change & Thorough change management system, see \emph{Change Management} \\
		Ambiguous requirements & ROFL! \\
		\bottomrule
	\end{tabular}
	\vspace{1.5em}
	\caption{Risk mitigation and management}
	\label{tab:rmm}
\end{table*}

The final stage of the risk management process is monitoring. Throughout the duration of
the project each identified risk will be reassessed for changes to its probability and
impact. This allows mitigation and management strategies to be revisited to ensure that
they are as effective as possible.

The remainder of this section describes with some the foreseeable challenges that have been
identified. These are less quantifiable than the risks detailed in the table above.

\subsection{Time constraints}

Game development projects are famous for scheduling issues that threaten to delay the
release of a product. Developers often find themselves facing ``crunch time", a period
of extreme work overload, in an effort to deliver a game on time.\cite[-1em]{groen2011}
A survey of problems encountered in game development performed by Petrillo et al. found
that two of the most common issues are missing deadlines and crunch time that results 
from this.\cite[1em]{petrillo2009} Although delays are a challenge common to all projects,
the survey found that the need for multiple disciplines working together (programming,
graphic design and music composition for example) to create a quality game causes
deadline problems to occur even more frequently. These common problems have their roots
in the time constraints imposed on a particular project.

This project has approximately twenty five weeks in which to develop a fully functioning
game that meets the requirements specified previously. This is a relatively short amount
of time in which to deliver a complex game. By adhering to the project management
and software development techniques laid out previously it is hoped that the project
can be kept on schedule and the final deliverable be released on time and to specification.

In his essays on software development, Frederick Brooks argues that the complex
communication structures in a team is a major cause of delays to software projects.\cite{brooks1995}
Fortunately, this project is run by a small team of four and so should find that
communication overhead is less of a problem. The problem of bringing any new team
members up to speed can also be ignored since this is a static team.
However, the short time frame available for completing the project is still a
major challenge to be overcome.

\subsection{Writing a successful AI}

Artificial intelligence can often be a make or break factor in determining the success of
a game.\citepage{rabin2002}{page 3} Without a convincing intelligence system, a game can
quickly become infuriating to play. This is because a human player expects any computer
controlled components to behave sensibly. In some cases well known algorithms exist that
enable `intelligent' behaviour to be implemented relatively easily, for example the use
of the A* search algorithm for pathfinding. However, higher level intelligence systems
are much more challenging. A system capable of creating and executing quality plans
from abstract orders is going to be one of the hardest components to implement.

As well as providing an entertaining experience an AI system must also be efficient.
There cannot be large delays between the user giving an order and it being carried
out. Any planning algorithms have to run quickly otherwise the lag in feedback will
detract from the realism of the game. An inefficient AI system could also stop the game
from running smoothly --- which is of great importance for a real-time strategy game.
This would lead to a poor user experience causing people to stop playing the game.

\subsection{Efficiency problems}

\subsection{Minimal graphics libraries available}

Some investigation into the Haskell graphics libraries available has already been undertaken.
The Gloss package has been identified as a suitable candidate because it exposes a clean
functional API and hides away the details of OpenGL. Unfortunately it is a relatively simple
library and does not provide some required features such as windowing and clipping.
