\section{Stakeholders and Communication Plan}
\label{section:communication}

The various parties identified as stakeholders are shown in Table \ref{tab:stakeholders} (overleaf). The relationship between the stakeholder and the project is shown, along with a rough estimate of their power and interest.\sidenote[][-2em]{See \bibentry{mendelow1991stakeholder}} This grid will form a reference for making sure that all interested parties are communicated with appropriately throughout the duration of the project.

\vspace{1em}

\begin{table*}
	\small
	\renewcommand{\arraystretch}{1.9}
	\begin{tabular}{p{9em} p{5em} p{2.5em} p{2.5em} p{9em} p{9em} p{8em}}
		\toprule
		\emph{Stakeholder} & \emph{Relationship} & \emph{Power} & \emph{Interest} & \emph{Requirements} & \emph{Measurements} & \emph{Communication Strategy} \\
		\midrule
		
		Project Team & Internal & High & High & 
		Good working environment, creative input. & 
		Meeting project spec, good grades! & 
		Various, detailed elsewhere. \\
		
		Supervisor --- Sara Kalvala & Internal & High & High & 
		Good communication. & 
		Adherence to spec, good PM, high quality write-up. & 
		Weekly meetings. \\
		
		Client --- Matt Leeke & Core \mbox{External} & High & High & 
		Good communication, creative input, hard work & 
		Strength of software, strength of report & 
		Weekly meetings. \\
		
		Second Assessor & Core \mbox{External} & High & Low & 
		None & 
		Marking scheme & 
		Deliverables only. \\
		
		Projects Organiser --- Steve Matthews & External & High & Low & 
		Cooperation when required. & 
		Deliverables on time. & 
		Email or meeting if required. \\
		
		Playtesters & External & Low & High & 
		Able to report issues / feature requests. & 
		Strength of game, input considered. & 
		Email. \\
		
		Other future users & Rest of World & Low & High & 
		Game works and is reliable. & 
		Strength of game, re-playability. & 
		Website, forums, blog. \\
		
		The Haskell and FP Communities & Rest of World & Low & High & 
		None & 
		Interest in  / strength of results and tools released. & 
		Online as above, and via the final report. \\
		\bottomrule
	\end{tabular}
	\vspace{1.5em}
	\caption{Stakeholders for the project.}
	\label{tab:stakeholders}
\end{table*}

\noindent Communication within the project team is examined in detail elsewhere in this document, so the remainder of this section is concerned with the other stakeholders.

\subsection{Supervisor Meetings}

Regular communication with the project supervisor is likely to be a critical factor in success of the project. For this reason a weekly meeting with at least one member of the group if not more will be high priority.

\subsection{Client Meetings}

The client is clearly vital to the success of the project, and continual feedback on each release will allow for early identification of any problems. At least one meeting per release (ie each week) will be required, as well as further meetings and correspondence as needed.

\subsection{Projects Organiser and Second Assessor}

The projects organiser could exert a strong influence over the project if they wished, but as there are many projects and it would be inappropriate for them to demonstrate partiality, extended levels of communication are unlikely to be necessary. Brief updates pertaining to deliverables is all that should be required. But if the project organiser initiates communication then they should be made a high priority.

Communication with the second accessor is, for the most part, not appropriate, excepting when within the remit of the deliverables, i.e. the report and presentation themselves.

\subsection{Playtesters and End Users}

\subsection{The Haskell and Functional Programming Communities}

The overall end goal of the project is not just a game, but an examination of Haskell and Functional Programming as a game development environment. 


