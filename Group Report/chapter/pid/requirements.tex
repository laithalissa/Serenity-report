\section[Requirements]{Requirements}
\label{section:requirements}



% help site at http://www.projectmanagementhelp.com/how-to-write-functional-requirements/
% bullet point these requirements, describe them, specify any details.
% include bain quite as footnote
\subsection{Functional Requirements}

% what it is 
% what it will be used for
% how we will measure success

\begin{description}

\item[AI] 
The AI, artificial Intelligence will include any algorithms under the field of Artificial Intelligence, such as planning.
The Player's fleet will be controlled by an AI algorim. It will use a planning algorithm that takes a high level objective given by the player, and generates a series of steps to achieve that objective.
The measurement of success for AI is being able to give an order to a ship that requires at least 3 sub steps to achieve that goal, and the AI successfully completing that goal(assuming it is possible to achieve this goal).
When an AI is given a goal that is either impossible to achieve given the world state, or is later invalidated due to world changing before the goal is achieved, the goal will be canceled, and the AI will wait for the next goal assigned by the player.
The AI will be required to control single ships within the fleet, never the fleet itself, since a human player is always in control of a fleet.

\item[Realtime Strategy]


\item[Multiplayer]
Multiplayer is a game which two human players can participate in, interacting with each other within the game world.
For this specific game, both players will be operating different computers on the same Network.
The network requirements are that it is a Local Area Network, allowing much greating bandwidths than the World Wide Web.

\item[Campaign]
The gameplay will consist of a fixed number of battles between the two players, will a penultimate final battle to determine the winner.
The winner of each battle, will be granted a bonus in the final battle, giving them an advantage against their opponent.
At the start of a campaign, both players will choose their fleet with ship customization options.
The fleet the player chooses is final and cannot be changed until the capaign is over.
Between each battle of th campaign, the player is allowed to customize their fleet.


\item[Ship Customization]
Ships will have two types of slots: System Slot, and Weapon Slot.
The number of each slot is specific to that class of ship.
System slots hold internal systems to the ship that are optional, ie the system is not required by the ship to function, but provides additional functionality or benefts.
All system slots are the same size, hence any system can fit in any system slot.
A weapon slot allows a weapon to be fitted to that slot.
Each weapon slot can can at maximum one weapon equipped, that must belong to the correct weapon class to fit the weapon slot.
Weapon slots can be empty.
Each ship will belong to a specific class of ship which specifies the hull size, the number of system/weapon slots, and their positions on the ship.
When a player is choosing their fleet, they will have a fleet budget constraint, such that the cost of their fleet must less than or equal to this fleet budget.
Each ship class will have a different fleet cost, with bigger ship hulls costing more than smaller ones.
different weapons/systems will have different costs, hence it better weapons/systems will have a greater cost than less weapons/systems.


\item[Resource System]
Three resources exist within game: Fuel, Anti-Matter, and Metal.
These resources are only used by the ships, for different purposes.
The resources only exist within a battle of a campaign, they are not persistant throughout a campaign.
Fuel maintains the ship's shields, without fuel, the ship will have no shields, and any damage will be directed to the ship's hull.
Anti-Matter is used to power certain weapons on the ship. These weapons are very powerfull, however they use large amounts of Anti-Matter to fire.
Metal is used to repair ship's hulls after being damaged.
Each ship will have their own supply of each resource, however they cannot generate these resources, so when that ship's supply of resource runs out, the ship is unable to gain that resource's beneft.
Resources are located on planets of the map.
Resources are gained by capturing planets with that resource.
When a planet is captured, it generate a constant supply of resources to the player that captured it.
The quantity of each resource the planet generate is specific to that planet, and can vary dramatically between planets.
The resources that are generated by the player's planets are added to that player's stockpile.
The player's ships will resupply from the player's stockpile when it is at any planet owned by the player.


\item[Ship Destruction]




% \item[Gameplay]
% mention ship destruction and resources that drained during battle, not loss of ships


\item[Planetary Capture]
Capturing planets is the only way of gaining resources.
A planet can belong to at most one player.
To capture a planet, the player must have ships at the planet for a specific amount of time.
if the planet belongs to another player, it will take twice as long to capture the planet, than if the planet was not owned by any player.
A planet can only be captured if their are no other player's ships at that planet.


\item[Tactical Zoom]


\item[Fog Of War]



\item[Operating System Requirements]
Mac OS or Linux(kernel 2.6 or later) 

\item[Haskell]

\end{description}

functional - 2d, realtime strategy, multiplayer, ship design, resource system, ai, planetary capture resource system, possiblity of campaign style multiplayer, tactical zoom/gameplay, fow, hw requirements, haskell.


\subsection{Non-Functional Requirements}

\begin{description}

	\item[Fun]

	\item[Short lived game sessions]

	\item[Reliable]

		Both the client and server should be stable programs that are not prone to
		crashing. If either were to crash regularly then it would ruin the experience
		and cause people to stop playing the game.

		The networking component should also be reliable. Minor network disruption should
		not cause a huge loss in communication between the clients and server.

	\item[Secure]

		Although the game server will initially be intended for LAN usage it is
		important that it should not cause a computer running it on the Internet to be
		exploitable. It should not be vulnerable to attacks such as denial of service,
		which would stop the machine from performing any other tasks whilst under attack,
		or remote code execution, which could allow an attacker to take control of the
		target machine.

\end{description}

