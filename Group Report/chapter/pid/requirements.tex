\section[Requirements]{Requirements}
\label{section:requirements}



% help site at http://www.projectmanagementhelp.com/how-to-write-functional-requirements/
% bullet point these requirements, describe them, specify any details.
% include bain quite as footnote
\subsection{Functional Requirements}

% what it is 
% what it will be used for
% how we will measure success

\begin{description}

	\item[Realtime Strategy]


	\item[Multiplayer]

	As a multiplayer game, two players must be able to participate in the same battle. These
	players will be on different computers on the same network. The game must be capable of running
	smoothly on a local area network, which experiences less disruption than the Internet.

	\item[Ship Customisation]

	The ships that make up a player's fleet will each have a number of pluggable slots.
	There will be two types of slot: system slots and weapons slots. System slots allow for extra
	internal systems to be added to the ship, for example extra shields or long range scanners.
	Weapons slots allow the player to choose which types of weapons their ships will use. However,
	a fleet budget constraint will prevent a user from exclusively using the best weapons on
	every ship.

	\item[Resource System]

	Three resources exist within the game: fuel, metal, and anti-matter. These resources are only
	used by the ships. Fuel maintains a ship's shields, without a shield any damage will directed
	to the ship's hull. Metal is used to slowly repair a ship's hull after it has been damaged.
	Anti-matter is a rare resource that is used by the most powerful weapons available.

	Planets generate a constant supply of resources, so players can gain extra resources via
	planetary capture. The resources generated by the planets owned by a player feed into that
	player's global stockpile. Individual ships then draw resources from this stockpile.

	% Do ships get resources from the stockpile automatically, or do they have to visit planets?

	\item[Planetary Capture]

	Planet ownership is the only method of generating resources. Planets can only belong to
	one player at a time at most. To capture a planet, a player must have their ships in control
	of the planet for a certain period of time. If the planet belongs to the enemy then it will
	take twice as long for it to be captured than an unoccupied planet.

	\item[Tactical Zoom]


	\item[Fog of War]

	Players must not be able to see the state of the entire map unless they control it all. There
	will be three levels to the fog of war: unknown, visited, and visible. Any locations on the map
	that a player's ships have not visited will be `unknown' and the player will not be able to see
	anything that is at that location. Any locations that have been explored at least once will be
	`visited'; the player will be able to see the general layout of the area, but not any details
	such as enemy ships. Finally, any locations covered by planets and ships owned by the player will
	be `visible' and all aspects of the map in the area will be revealed to the player.

	\item[AI]

	A sophisticated artificial intelligence system will be an important component in the game.
	Each player's fleet will be controlled through an AI system. It will use a planning algorithm
	that takes a high level objective, given by the player, to generate a series of steps to
	achieve that objective. The AI must control the individual ships that make up the fleet; it
	is up to the human player to decide on the overall strategy of the fleet.

	A successful AI system must receive orders and quickly convert them into a sensible series of
	actions which it performs autonomously. If an impossible goal is set, or an existing goal is
	invalidated by changes to the world, then the AI must detect this and stop acting on it.

	% Basic AI elements, e.g. pathfinding

	\item[Campaign]

	A campaign mode will be available which consists of a fixed number of battles between the two players.
	The final battle will be the `showdown' that determines the overall winner. The victor of each of
	the earlier battles will be granted bonuses toward the final battle, giving them an advantage against
	their opponent.

	Between every battle each player will have an opportunity to perform minor customisations on
	the ships in their fleet.

	\item[Operating System Requirements]

	The game must be playable on recent versions of Mac OS X and Linux.

% \item[Gameplay]
% mention ship destruction and resources that drained during battle, not loss of ships

\end{description}

\subsection{Non-Functional Requirements}

\begin{description}

	\item[Fun]

	\item[Short lived game sessions]

	\item[Reliable]

		Both the client and server should be stable programs that are not prone to
		crashing. If either were to crash regularly then it would ruin the experience
		and cause people to stop playing the game.

		The networking component should also be reliable. Minor network disruption should
		not cause a huge loss in communication between the clients and server.

	\item[Secure]

		Although the game server will initially be intended for LAN usage it is
		important that it should not cause a computer running it on the Internet to be
		exploitable. It should not be vulnerable to attacks such as denial of service,
		which would stop the machine from performing any other tasks whilst under attack,
		or remote code execution, which could allow an attacker to take control of the
		target machine.

\end{description}

