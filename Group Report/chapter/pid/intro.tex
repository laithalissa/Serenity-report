\chapter[Introduction]{Introduction}
\label{ch:motivation}

\chapterepigraph{If you aren't sure which way to do something, then do it both ways and see which works better.}{John Carmack}

\newthought{Functional programming} (FP) has a long history, with its roots in the $\lambda$-calculus of Alonzo Church.\citefix[-1.5em]{church1932} One of the first functional programming languages was Lisp, invented by John McCarthy in 1958, which is still used today, over 50 years later.\citepage{reilly2003}{pages 156--157} Various languages have refined and extended the functional paradigm over the years --- probably the most notable as of now being Haskell, Scala, OCaml, F\#, and Erlang.

Despite the amount of time such languages have been available, use in industry has typically been far less than that of languages such as C, C++, and Java.\citepage[-2em]{odersky2010programming}{page 11} That being said, in recent years there has been increasing use of functional techniques and languages in certain areas. Erlang was designed for the development of highly fault tolerant telecommunication systems.\cite[-1em]{armstrong2007history} OCaml is used extensively by some organisations in the financial sector to create trading algorithms and other similar applications.\citefix[1em]{minsky2011ocaml} Scala is also increasingly popular, helped in part by its compatibility with the Java Virtual Machine (JVM) and object oriented design.

One of the often cited reasons against the use of functional programming in some domains is that of performance. This is due in part to mutable data structures generally being easier to represent on machine hardware; and it therefore being harder for functional compilers to convert the code into an efficient representation.\citefix{paulson1996ml} However, it is not a given that any program would run slower if written in a functional language: in some cases lazy-evaluation or compiler optimisations made possible by immutability can mean a program runs faster, plus advanced compiler techniques such as array fusion can lead to programs nearing the efficiency of hand-crafted C. And with modern machines getting ever faster, the domain of problems that require high levels of efficiency is getting smaller.

Performance problems alone cannot account for the fringe position of functional programming. The efficacy of the functional approach has been touted for many years,\sidenote{For example see \bibentry{hughes1989functional}.} yet it is still rare for mainstream projects to make any use of functional languages.

Instead of researching and discussing the theoretical advantages of the functional paradigm, this project will attempt to demonstrate the value of functional programming by utilising it in a problem domain that should pose a significant challenge that is not normally considered a `good' domain for FP. The chosen application for the project is a game.

\section{Why a Game?}

Game programming brings together a diverse range of computing areas. For example human interaction in real time, detailed graphics and animation, artificial intelligence / planning, networking, and various other dynamic elements.\sidenote{See \bibentry{crawford1984art}.} A game is also a tangible, sizeable piece of software, yet achievable for a four person group over two terms.

As well as demonstrating FP over a wide range of areas, a game also represents a serious business venture.\cite{essentialFacts2012} Computer games have been a huge industry for almost as long as personal computers have existed. Demand is high, and a vast amount of games are being continually developed, from triple-A ventures and big companies, down to indie companies and fan groups.

This project will certainly not be the first game ever to be developed in a functional language, nor is it likely to be the last. The project shall therefore not only deliver the finished game, but fully document the process --- explaining what went well and how the functional approach benefited the construction, as well as what proved challenging.

\section{Picking the Language}

There are several functional languages that would be suitable for this project, most of which have already been mentioned. Of these, the language chosen is Haskell. The reasons for this are outlined below.

\begin{description}
	\item[Concision] Haskell code is concise, yet readable. This is a very real advantage as it allows both for fast writing of code, as well as fast refactoring and maintenance. 
	\item[Purity] Haskell is a pure functional language, with all the advantages that gives. However side-effects and mutability are needed for real programs (especially games) and Haskell has excellent tools for solving these problems, via the IO Monad, State monads, etc. The `do' syntactic sugar makes Haskell one of the best languages for this.
	\item[Speed] Haskell has very good compiler support, and the Glasgow Haskell Compiler (GHC) is capable of producing highly efficient code. 
	\item[Type System] Haskell has a very advanced type system, which makes bugs and errors in refactoring easy to detect quickly. Automatic type inference allows for these advantages without the type system slowing down the programmer.
	\item[Testing] The purity of Haskell enables automated testing techniques not possible even in other functional languages. There are also well supported testing libraries available for both pure and impure code.
	\item[Community and Library Support] The Haskell community is very active and there are extensive libraries available for it. Due to Haskell compiling to C, most C system libraries have Haskell bindings. There are OpenGL and OpenAL, for example.
	\item[Familiarity] All members of the group have some experience with Haskell, and consider developing with it to be very enjoyable.
\end{description}

\section{Existing Systems}

To avoid confusion we shall consider separately research into functional programming for games, and what game we will actually make given current and historic trends in gaming.

\subsection{Existing research into Functional Programming of Games}

One of the most well known games written in a functional language (Haskell, as it happens) is \emph{Raincat}\sidenote{Source available online from \url{raincat.bysusanlin.com}.} written in Haskell and developed by Carnegie Mellon students in 2008. There is also a game company, \emph{ipwn studios}, who exclusively use Haskell for their products.\sidenote{See there website: \url{ipwnstudios.com}.} Despite this, there is little work on the academic research side that supports or opposes functional languages for games. 

There was a similar project in 2005 by Mun Hon Cheong, an undergraduate at the University of New South Wales;\cite{cheong2005functional} and though Cheong did manage to create a complete 3D game and gave detailed descriptions of some of the code techniques, the project did not provide a detailed insight into what exactly was effective, or challenging, about the use of a functional language.

The fact that an exhaustive search of the literature in this area revealed only a single undergraduate dissertation highlights the paucity of available research in this area, and coupled with the growing interest in functional languages for game development shows the case for this project.

\subsection{Existing Games}

Needless to say, the complete history of gaming, even if only restricted to computer games, would be too lengthy to examine here. The gaming industry has grown hugely since the early commercial computer game systems, in parallel with the huge developments in the capabilities of computers themselves. And while some games today are magnificent technical achievements with breathtakingly detailed graphics, sounds, and physics engines, there are many popular titles from indie game companies using only 2D graphics and simple engines enjoying success. It is not just the cutting edge of technology that can make a game fun.\sidenote{This is a complex issue. For a more complete treatment see \bibentry{malone1981makes}.}

In order to serve the purposes of the project, it will be necessary to create a game designed to compete in the current market. This doesn't mean it has to be as complete, complex, or technically sophisticated as a triple-A title, but it does have to be a game that, if fleshed out fully, would be considered fun and suitable for a small company or similar organisation to sell. Without this any conclusions drawn about the efficacy of the approach will not be sufficiently valid to developers.

For this reason it is worth briefly reviewing a few games --- some modern, some less so --- in order to identify what would be an appropriate brief for a game to help achieve the project's aims.

An early game that was hugely successful, as well as controversial, is \emph{Doom}, a first person shooter developed by John Carmack and John Romero of id Software and released in 1993. Doom was marketed using a shareware model --- the first third of the game being distributed for free, and the rest available for purchase. Doom represented a revolution in what was possible in a computer game, and is widely accepted as the game that popularised the first-person genre. The slickness of the graphics engine, the thought provoking levels and puzzles, the controversial satanic imagery, all contributed to Doom's success. But the arguably the greatest innovation with the most effect on future gameplay was its multiplayer modes. Over modem or local serial connections, players could join forces to complete the main game, but could also battle each other in violent showdowns, coined \emph{deathmatches} by Romero. The name stuck.

Even though Doom is now very old, and the graphics looks hugely out of date, it is still relevant to consider just how successful the multiplayer model in Doom was and still is. Battles are short, skilful, and highly addictive. Carmack and Romero predicted that Doom would become the number one cause of non-work in offices across the world, and they were right. Especially given the limited time constraints of the project, and the amount of writing time single player plot lines can involve, a compelling multiplayer mechanic would be a sound basis for the new game.

A game hugely influential in the realm of real-time strategy is \emph{Total Annihilation} (TA), released by by Cavedog Entertainment in 1997. The reception to TA was extremely positive, and the game is still actively played to this day. TA was notable for an advanced resource system that required careful balancing, and 

% Games
% v Gratuitous Space Battles
% x Sins of a Solar Empire
% x Faster Than Light
% v Total Annihilation
% x Mech Commander 2

% likes:
%    - resources allocation to different systems
%    - micro management
%    - upgrading ships by buying weapons
%    - the details of battle effect the overall campaign (repairing damaged parts costs money)
%    - randomly generated campaign (x sectors, then final showdown)
% not likes: 
%    - game play can be stale dependening on ship layout, ie slow guns
%    - resource allocation rather static during battle, changes between battles and start of battle, but not during

Faster Than Light, a turn based strategy game with real time strategy battles, is set in space, with the player taking command of a ship with the objective of getting to the other side of the galaxy.
The ship has an upgrade system that allows the player to upgrade various systems/parts of the ship giving advantages in the next battle.
This upgrade system gives the player an incentive to battle, since winning a battle grants currency.
The ship has energy that needs to be allocated to the various systems such as shields, weapons, and life support.
This gives the player a greater control over their ship, whilst allowing them to tweak the capabilitities of the ship during battle, ie temporarilly dropping life support to boost their weapons.
During battle, the player's responsibilitities can vary completely from having to do nothing to having to pause the game every few seconds to calculate the next optimal move. Having a varied level of responsibility adds to the dynamic gameplay, however this game varied too much, ranging from complete bordom to a single battle taking much longer than it should.
A multiplayer mode was missing from the game, causing the the gameplay to become predictable, and not replayable.

% likes
%    - planets would become battlenecks causing the majority of battles to occur their.
%    - different play styles existed where their would be 
% dislikes
%    - game would run way too long
%    - resources would only be used for building, and wouldn't inflence a battle
% dislikes

Sins a of Solar Empire was a futeristic real time strategy based in space, with the game world modelling a graph of planets, where the player could only move ships between planets that were connected. 
The objective was to wipe out the opposing faction(s) by destroying their ships and capturing their planets.
The gameplay didn't have many twists to the outcome of a batlte, resulting in the dominant player continuing to graduly take ground, resuling in very long and boring gameplay.
Due to the layout of the world, certain planets would become bottleneck where the majority of battles occured. 
A race would ensure to capture these planets that would become the bottlencks, adding to the player's overall strategy.
The downside of this was that it was appearent who would likly win based on who had captured these bottleneck planets.
The layout of each game was precedurally generated, making each game unique and greatly improving the replyability factor for the game.
A resource system existed that had 3 resources: Credits, Metal, and Crystal. 
These resources were only used for the building of fortifications and ships, and would not effect the outcome of the battle directly.


\section{Methodology}

\begin{itemize}\itemsep-3pt
	\item Expand on documenting develop process of game in Haskell
	\item Project should simulate making a game that would be competitive in the current indie game market
	\item The need to do literature reviews / case studies into FP, games, and games written in FP.
	\item Outline rough parts of schedule in relation to this.
\end{itemize}


\section{Legal, Ethical, and Social Issues}
\label{section:professional_issues}

% legal
One potential legal issue faced by this project is the use of third party software.
It must be ensured that any third party libraries included in the code are licensed
appropriately. This means only using software with a permissive license (e.g. Apache, BSD, or MIT licenses) and no proprietary software.

Game publishers such as Electronic Arts, and Lion Head, consider their games as intelligectual property, copywriting their games.
It is infeasable to check that all previous games published don't bear great similarities which would result in a court case.

% ethical


- gaming addiction
  - campaigns will be 5 battles long, allowing players an escape after 5 battles.

- dependency on whales (0.4\% of Zynga's customers account for 80\% of it's  1 billion profits.



% social issues
- language game in: english
- racism
- vialence





