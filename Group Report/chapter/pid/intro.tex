\chapter[Motivation]{Motivation}
\label{ch:motivation}

\chapterepigraph{If you aren't sure which way to do something, then do it both ways and see which works better.}{John Carmack}

\newthought{Functional programming} has a long history, with its roots in the $\lambda$-calculus of Alonzo Church.\citefix[-1.5em]{church1932} One of the first functional programming languages was Lisp, invented by John McCarthy in the 1958; and Lisp is still used today, over 50 years later.\citepage{reilly2003}{pages 156--157} Various languages have refined and extended the paradigm over the years --- probably the most notable as of now being Haskell, Scala, OCaml, F\#, and Erlang.

Despite the amount of time such languages have been available, use in industry has typically been far less than that of languages such as C, C++, and Java.\citepage[-2em]{odersky2010programming}{page 11} That being said, in recent years there has been increasing use of functional techniques and languages in certain areas. Erlang was designed for the development of highly fault tolerant telecommunication systems.\cite[-1em]{armstrong2007history} OCaml is used in the financial sector to create trading algorithms.\citefix[1em]{minsky2011ocaml} Scala is increasingly popular, helped 

One of the often cited reasons against the use of functional programming in some domains is that of performance. This is due in part to mutable data structures generally being easier to represent on machine hardware; and it therefore being harder for functional compilers to convert the code into an efficient representation.\citefix{paulson1996ml} However, it is not a given that any program would run slower if written in a functional language: in some cases lazy-evaluation or compiler optimisations made possible by immutability can mean a program runs faster, plus advanced compiler techniques such as array fusion can lead to programs nearing the efficiency of hand-crafted C. And with modern machines getting ever faster, the domain of problems that require high levels of efficiency is getting smaller.

Performance problems alone cannot account for the fringe position of functional programming. The efficacy and advantages of the functional approach to programming and problem solving have been stated many times over a considerable number of years,\citefix{hughes1989functional} yet it is still rare for mainstream projects to make any use of functional languages or tools.

Instead of researching and discussing the theoretical advantages of the functional paradigm, this project instead will attempt to demonstrate the value of functional programming by utilising it in a problem domain that should pose a significant  challenge, that is not normally considered a `good' domain for FP. 

\section{Why a Game?}

Game programming brings together a diverse range of computing areas. A game involves human interaction in real time, detailed graphics and animation, artificial intelligence / planning, networking, and various other dynamic elements. A game is also 

\section{Why Haskell?}