\chapter[Project Management]{Project Management}
\label{ch:management}

\chapterepigraph{``All things are created twice; first mentally; then physically.  The key to creativity is to begin with the end in mind, with a vision and a blue print of the desired result."}{ Stephen Covey}

\newthought{Project management} is often incorrectly considered to be the coordination of steps a project must move through to be completed on time, the real challenge of project management is constantly incorporating the requirements of the customer into an ever changing system to ensure that the progression of the project is always in the customers interest. A major concern of project management is to minimise wasted time and resources without compromising the quality of the project, wasting time on this project would result in a poorer quality product.

\section{Risk Management}
\label{section:risk}
 
A proactive approach to risk management was taken. This technique was chosen in order
to maximise the probability of avoiding risks instead of having to move into `fire-fighting mode'
if something went wrong.\citepage{pressman2010}{page 745}
 
As part of this proactive risk management strategy, a number of potential risks were identified.
These risks are shown in table~\ref{tab:risks} along with their estimated probabilities of occurring
and impact if they were to occur.
 
\begin{table*}
	\small
	\begin{tabular}{l p{\textwidth / 2} l l}
		\toprule
		\emph{Risk} & \emph{Description} & \emph{Probability} & \emph{Impact} \\
		\midrule
		Length underestimate & The time required to develop the software is underestimated & Medium & High \\
		Team member illness & One or more team members unable to work due to illness & Medium & High \\
		Hardware failure & Damage to critical hardware causing loss of data & Medium & Medium \\
		Size underestimate & The size of the deliverable has been underestimated & Medium & Medium \\
		Requirements change & Large number of changes to requirements during development & Low & Medium \\
		Ambiguous requirements & Requirements are not fully understood or misinterpreted leading to
			loss of development time as the specification is recreated & Low & Medium \\
		\bottomrule
	\end{tabular}
	\vspace{1.5em}
	\caption{Risk identification and analysis.}
	\label{tab:risks}
\end{table*}
 
With the risks identified, and their likelihood and consequences estimated it is necessary
to draw up plans to mitigate their effects. There are three types of management strategies 
for individual risks: avoidance strategies to reduce the probability of the risk occurring;
minimisation strategies to reduce the impact of the risk; and contingency plans to deal with
the risk if it does arise.\citepage{sommerville2011}{page 601} It is best to avoid the risk,
but if this is not possible then minimisation of the effects and, finally, contingency plans
should reduce the overall impact of a risk on the project. The mitigation and management
strategies for each risk previously identified are listed in table~\ref{tab:rmm}.
 
\begin{table*}
	\small
	\begin{tabular}{l p{37em}}
		\toprule
		\emph{Risk} & \emph{Mitigation / Management} \\
		\midrule
		Length underestimate & Detailed work breakdown with weekly releases to ensure that
			schedule slippage can be caught early \\
		Team member illness & Well documented code (enforced by the software librarian) so
			that other members can quickly start work on less familiar sections of
			the codebase \\
		Hardware failure & Backups and distributed source control, see section~\ref{section:tools} \\
		Size underestimate & Detailed work breakdown structure \\
		Requirements change & Thorough change management system, see section~\ref{section:control} \\
		Ambiguous requirements & Thorough planning phase \\
		\bottomrule
	\end{tabular}
	\vspace{1.5em}
	\caption{Risk mitigation and management.}
	\label{tab:rmm}
\end{table*}
 
The final stage of the risk management process is monitoring. Throughout the duration of
the project each identified risk was reassessed for changes to its probability and
impact. This allowed the mitigation and management strategies to be revisited to ensure that
they were as effective as possible.
 
The two previously identified risks that actually occurred were team member illness and length underestimate.
On a couple of occasions a team member was ill and unable to attend group work sessions or work to their full
capacity. Fortunately, the team was able to reduce the impact of this by ensuring that the work each individual
was performing was not hindered by an absence. This was done by allocating work tasks to be as separate as
possible to allow more parallel development to occur. Also, an illness was never so severe as to stop a team
member from working for longer than a day or two. However, the issue of length underestimation was more serious.
 
% Length underestimation
%
% 1. More time required for network and GUI than hoped
% 2. (1) slowed the development of the ideal game
% 3. (1) was very informative for our goal of investigation of Haskell for game development
 
The proactive approach to risk management was a good choice. By reviewing the potential
risks before starting the development phase of the project it was much easier to avoid risks that could
have had disastrous consequences for the project. For example, by implementing a thorough backup strategy
prior to any data loss actually taking place it was ensured that no work would have been lost if a hardware
failure had occurred. Continuously monitoring and reassessing these risks was also helpful in preventing
any risks becoming more probable or having a greater impact.

\section{Grievance Policy}
It's imperative to document a grievance guideline to ensure that any grievance procedures are fair and impartial. %ref gov
 Because the developer team is small, an internal dispute would have a high cost to the project, so grievance issues need to be dealt with promptly. 
  
\begin{enumerate}[i)]
    \item{Attempt to resolve the grievance issue informally by the team manager.}
    \item{If the issue cannot be resolved informally, and affects the entire project group, then address the issue at the next meeting and attempt to find a resolution in a group environment.}
    \item{If the issue cannot be solved by a formal group meeting then a managerial confrontation is required to reduce the impact on the project.}
    \item{If the issue still cannot be resolved then a complaint should be made to the module supervisor and university procedure should be followed from then on.}
\end{enumerate}

\section{Agile Methodology and Weekly Releases}
The project team has adopted the agile development methodology, the project is broken down into smaller increments which require minimal planning and are unlikely to require long term consideration. Development iterations allow the team to regularly deliver working software, reducing the likelihood of delays going unnoticed and becoming obstructions at a later stage. The development cycle the team has selected requires a component release every Tuesday evening, making each iteration span exactly one week. Each release should include the latest completed iteration of the project, the first notable release will be the game prototype which is scheduled for release on 27th November 2012, further releases pushing the game through beta and eventually arriving at a release candidate by 19th March 2013.
%%
%% 19th = Tuesday AFTER term ends- feel free to change that if it's too ambitious :P
%% term dates 
%% http://www2.warwick.ac.uk/study/termdates/
%%
The team has readily adopted pair programming, as well as test-driven development and continuous integration to help maintain code stability through multiple iterations, techniques which are strong advocates of the agile development methodology.

There are aspects to development cycles beyond the milestones and time scheduling, programmer welfare is a large focus of agile development strategies. Ensuring programmers are not overworked or lose interest in the project is an emergent factor in maintaining quality in software projects, and although the team and project managers cannot realistically limit the hours a team member spends programming over a week (mainly due to other programming assignments in parallel to the project), Wednesday afternoons and weekends have been avoided in the weekly timetable to guarantee personal time every week.


\subsection{Standup Meetings}
Standup meetings are held every week to discuss individual progress on the project, any issues an individual has been encountered, and what they will attempt to accomplish in the following week.

\section{Change Management}
As the project develops new ideas and approaches may become apparent, care must be taken to avoid blindly integrating changes into the original specification to protect the project from scope creep. The change management procedure involves assessing the viability and benefits of the change request, deciding whether the change would stop the project meeting the requirements, and if so, whether the customer and managers can reach an agreement which incorporates this change into the project or whether the change request should be rejected. Change viability is decided on the difference in cost and time required, whereas the requirements assessment is heavily dependant on the customer, and whether they think such a change would prevent the project meeting their requirements.

\begin{figure*}[h!]
	\includegraphics{"res/Change management diagram"}
	\caption{Change management workflow}
\end{figure*}

\section{Tools and Techniques}

\subsection{Source Control: Git}

\subsection{Tracking and Managing Releases: Trello}

\subsection{Bug Tracking: Fogbugz}

\subsection{Wiki}

\subsection{CI: Jenkins}

\subsection{Backups}

\subsection{Cabal}

\subsection{Whiteboard}
