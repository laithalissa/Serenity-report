\chapter[Project Management]{Project Management}
\label{ch:management}

\chapterepigraph{``All things are created twice; first mentally; then physically.  The key to creativity is to begin with the end in mind, with a vision and a blue print of the desired result."}{ Stephen Covey}

\newthought{Starting a sentence} with a new thought.

Our project team has adopted the agile development cycle as a basis for our management structure. We've broken the project down into critical components and have committed to weekly releases to ensure the project remains well scheduled throughout the course of the year.

\section{Risk Management}
Unexpected delays need to be anticipated and dealt with promptly. Although risks present themselves in many forms, the ultimate cost to the project is the loss of time. Falling behind schedule can prove disastrous to precariously timed projects, particularly those which have absolute deadlines. Our project schedule anticipate small delays by including feed buffers between components of the project which saves the (resource allocation) overhead of returning to work on archived components at a later stage. In addition to feed buffers our schedule has allocated a global project buffer--- a reserve of time allocated to optional components (such as extended gameplay modes) which can be reallocated to deal with major delays.

\section{Project Scheduling}

\section{Roles and Responsibilities}
\begin{description}
\item[Project Manager] Responsible for overseeing the project in general, including managing a schedule, organising meetings, collaboration sessions. Makes decisions involving tradeoffs between project cost, quality and particularly time. 

\item[Customer Liaison] Meets with the customer at regular intervals to discuss the project progress, outlook, and any issues which are require customer input.

\item[User Manager] Finds end users to test the product in the later stages of development, then provides feedback to the project manager and team leader. Prime responsibility is to identify issues which are not clearly visible from the project development perspective, but are more apparent to end users.

\item[Analyst] Responsible for ensuring the customer requirements are addressed during planning and development stages of the project, and that the solution addresses the customers needs sufficiently.

\item[Graphic Designer] Responsible for prototyping and developing graphical element designs, such as ship sprites as units.

\item[Software Librarian] Ensures team completes documentation to

\item[Line Manager] Overlooks day to day development, intervenes if a developer is off track, ensuring time isn't wasted perusing low priority work.

\item[Code Reviewer] Responsible for interpreting other developer's code, checking for logical inconsistencies and familiarising themselves with the project as a whole.

\item[Chairman] Responsible for coordinating meetings, ensuring all issues are resolved or at the very least discussed and all meeting participants have a chance to voice concerns and contributions.

\item[Testing and Integration Manager] Ensures the code is thoroughly tested for bugs, and any discovered bugs are flagged and dealt with promptly. 

\item[Security Officer] Checks for security flaws in the product, performs security evaluations (e.g penetration testing) to assert a reasonable level of security for the product.

\item[Team Leader] Responsible for keeping the project team
 
\item[Lead Developer] Consults other developers when they have difficulty, makes development decisions when major development issues arise and Enforces programming style and technique.

\item[Music Composer] Composes music for different parts of the game, must produce a product which the team leader is satisfied with.
\item[Tester] Performs general code testing (e.g. unit tests), generally developers are responsible for testing their own code, but additional testing is required after code integration to prevent the introduction of bugs caused by conflicting revisions.
\item[Programmer] Performs the day-to-day programming specified by the line manager, has implementation freedom provided the needs of the line manager ar met.
\item[Gameplay Design] Critically analysis gameplay design and experience, giving feedback to the team leader on how to improve the games's appeal.
\end{description}

\begin{description}
	\item{Laith Alissa}
	\item{Jon Cave}
	\item{Squid balls}
	\item{Vic Smith}
\end{description}

\section{Grievance Guidelines}
It's imperative to document a grievance guideline to ensure that any grievance procedures are fair and impartial. %ref gov
 Because the project team is small, an internal dispute would have a high cost to the project, so grievance issues need to be dealt with promptly. 
 
\begin{enumerate}[i)]
	\item{Attempt to resolve the grievance issue informally by the team manager.}
	\item{If the issue cannot be resolved informally, and affects the entire project group, then address the issue at the next meeting and attempt to find a resolution in a group environment.}
	\item{If the issue cannot be solved by a formal group meeting then a managerial confrontation is required to reduce the impact on the project.}
	\item{If the issue still cannot be resolved then a complaint should be made to the module supervisor and university procedure should be followed from then on.}
\end{enumerate}


\section{Requirements}
functional - 2d, realtime strategy, multiplayer, ship design, resource system, ai, planetary capture resource system, possibility of campaign style multiplayer, tactical zoom/gameplay, fow, hw requirements.
non-functional - fun, reliable, haskell, secure  

\section{Release Management}
As a motivating factor to keep the project on schedule we've committed to component releases every Tuesday evening. The release should include the latest completed iteration of the project, which is scheduled to develop as a prototype by the end of term 1.

\section{Develop}
\section{Status Meetings}
Standup meetings are held every week to discuss individual progress on the project, any issues an individual has been encountered, and what they will attempt to accomplish in the following week.
\section{Change Management}

Lorum ipsum dolor sit amet.