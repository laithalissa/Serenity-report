\section{Quality Controls}
\label{section:quality}

Quality controls are a set of methods that allow the product to be tested against the specification, identifying cases in which the product will not meet the specification.
Quality control can be automated for requirements that test the behaviour of the product, such as functional requirements. Non-functional requirements, which specify the the qualities that are required for the project to achieve a desired behaviour, are generally not quantifiable and therefore cannot be automated.

% what quality control is
% what it will do for us


% meets the specification
% bug free
% 

\subsection{Unit Testing}
Unit tests are used to test small, individual units of source code and ensure that the
code meets its intended design. Unit testing is a very useful practice because it helps
catch errors in code and allows a developer to refactor code by ensuring that it continues
to work as before.


\subsection{Component Testing}
Component testing is similar to unit testing, but focuses on larger pieces of the source
code; it is used to test the integration of a number of units. Component is useful to
ensure that the small units of code, which, through unit testing, are known to work
individually, work together correctly.

\subsection{Continuous Integration}
Unit tests are only really useful if they are run regularly. Doing so allows developers
to catch bugs as soon as they are introduced. This is useful because, as van Emden
and Moonen note, the cost of fixing a bug is much lower if it is discovered earlier in
the development cycle.\cite{emden2002} Therefore, a continuous integration server
will be used to perform a full build and test of the software every time a new change
is pushed to the source control repository.

\subsection{Playtesting}



\section{Success Measurement}
\label{section:success}

The various aims and goals of the project need to be measured for success individually to ensure that the project is successful overall. The following measures will be used.

\subsection{Stage Gate Model}

The project is organised into separate phases with defined control points, referred to as \emph{stages} and \emph{gates}.\sidenote{This is the stage gate model, see \bibentry{karlstrom2005combining}.} The gates ensure that each phase must be completed to an acceptable level of quality before the project can continue.

The precise measures used at each gate are described in the section \ref{section:quality}, but are mostly combinations of acceptance tests and feedback from playtesters.

\subsection{Acceptance Testing}

Acceptance testing is a tried and proved method for making sure project deliverables are of the required quality.\sidenote{See, for example, \bibentry{hsia1994behavior}.} A suite of tests might cover...

\subsection{Playtesting}
