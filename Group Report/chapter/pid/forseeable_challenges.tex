\chapter[Forseeable Challenges]{Forseeable Challenges}
\label{ch:forseeable_challenges}

\chapterepigraph{The lurking suspicion that something could be simplified is the world's richest source of rewarding challenges}{Edsger Dijkstra}

\newthought{Some kind} of introductory paragraph.

\section{Time constraints}

Game development projects are famous for scheduling issues that threaten to delay the
release of a product. Developers often find themselves facing ``crunch time", a period
of extreme work overload, in an effort to deliver a game on time.\cite[-1em]{groen2011}
A survey of problems encountered in game development performed by Petrillo et al. found
that two of the most common issues are missing deadlines and crunch time.\cite[1em]{petrillo2009}
Although delays are a challenge common to all projects, the survey found that the need for
multiple disciplines working together (programming, graphic design and music composition for example)
to create a quality game often caused problems. All of these common problems have
their roots in the time constraints imposed on a particular project.

This project has approximately twenty weeks in which to develop a fully functioning
game that meets the requirements specified previously.

\section{Writing a successful AI}

Artificial intelligence can often be a make or break factor in determining the success of
a game.\citepage{rabin2002}{page 3} Without a convincing intelligence system, a game can
quickly become infuriating to play.

However, an AI system must be efficient as well as providing an entertaining experience.

\section{Minimal graphics libraries available}

Some investigation into the Haskell graphics libraries available has already been undertaken.
The Gloss package has been identified as a suitable candidate. Internally it
uses OpenGL, but exposes a clean functional API. Unfortunately it is a relatively simple
library and does not provide some required features such as windowing and clipping.

\section{Efficiency problems}
