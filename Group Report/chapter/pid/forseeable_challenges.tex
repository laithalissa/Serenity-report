\chapter[Forseeable Challenges]{Forseeable Challenges}
\label{ch:forseeable_challenges}

\chapterepigraph{The lurking suspicion that something could be simplified is the world's richest source of rewarding challenges}{Edsger Dijkstra}

\newthought{Some kind} of introductory paragraph.

\section{Time constraints}

Game development projects are famous for scheduling issues that threaten to delay the
release of a product. Developers often find themselves facing ``crunch time", a period
of extreme work overload, in an effort to deliver a game on time.\cite[-1em]{groen2011}
A survey of problems encountered in game development performed by Petrillo et al. found
that two of the most common issues are missing deadlines and crunch time that results 
from this.\cite[1em]{petrillo2009} Although delays are a challenge common to all projects,
the survey found that the need for multiple disciplines working together (programming,
graphic design and music composition for example) to create a quality game causes
deadline problems to occur even more frequently. These common problems have their roots
in the time constraints imposed on a particular project.

This project has approximately twenty five weeks in which to develop a fully functioning
game that meets the requirements specified previously. This is a relatively short amount
of time in which to deliver a complex game. By adhering to the project management
and software development techniques laid out previously it is hoped that the project
can be kept on schedule and the final deliverable be released on time and to specification.

In his essays on software development, Frederick Brooks argues that the complex
communication structures in a team is a major cause of delays to software projects.\cite{brooks1995}
Fortunately, this project is run by a small team of four and so should find that
communication overhead is less of a problem. The problem of bringing any new team
members up to speed can also be ignored since this is a static team.
However, the short time frame available for completing the project is still a
major challenge to be overcome.

\section{Writing a successful AI}

Artificial intelligence can often be a make or break factor in determining the success of
a game.\citepage{rabin2002}{page 3} Without a convincing intelligence system, a game can
quickly become infuriating to play. This is because a human player expects any computer
controlled components to behave sensibly. In some cases well known algorithms exist that
enable `intelligent' behaviour to be implemented relatively easily, for example the use
of the A* search algorithm for pathfinding. However, higher level intelligence systems
are much more challenging. A system capable of creating and executing quality plans
from abstract orders is going to be one of the hardest components to implement.

As well as providing an entertaining experience an AI system must also be efficient.
There cannot be large delays between the user giving an order and it being carried
out. Any planning algorithms have to run quickly otherwise the lag in feedback will
detract from the realism of the game. An inefficient AI system could also stop the game
from running smoothly --- which is of great importance for a real-time strategy game.
This would lead to a poor user experience causing people to stop playing the game.

\section{Minimal graphics libraries available}

Some investigation into the Haskell graphics libraries available has already been undertaken.
The Gloss package has been identified as a suitable candidate. Internally it
uses OpenGL, but exposes a clean functional API. Unfortunately it is a relatively simple
library and does not provide some required features such as windowing and clipping.

\section{Efficiency problems}
