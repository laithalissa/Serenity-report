\section{Design Approach}

% how are we going about designing the game
% top down vs bottom up
% the advantages / disadvantages of both

Two methods were considered for our design approach: Top Down, and Bottom Up.
These are generic strategies for prioritising what to design first, both have their trade-offs, but can hybridised to suite the project at hand.

Bottom Up typically starts with existing software modules that are integrated together to achieve a grander system.
Since the existing softwares provide the needed functionality, the majority of the implementation stage is piecing these software products/modules togother to form a cohesion product.
This approach falls short, when the software has to be written from scratch, since its focus is more on integration of existing products.

Top Down is used to design a system from scratch.
It's focus is on simplicity, only breaking down one aspect of the design at a time, into its smaller sub components.
The design stage will continue to expand the subsystems of this design, until the subsystems begin to overlap with the implementation level, at this point, all further subsystem nodes are expanded at the implementation level.

A hybrid approach will be used between the two design philosophies, at the design level, the  entire product will designed and implemented by the team, however at the implementation level, third party software products will be decided upon before the implementation stage, and will need factoring in to the design.

When designing the game, key aspects of the game are designed first to meet the requirements of the game. For instance the AI used within the game, the role of the player, how multiplayer will work.
When design conflicts arise later down the timeline, the initial high priority decisions will take precedence, allowing the conflict to be resolved whilst not compromising the requirements of the game.




